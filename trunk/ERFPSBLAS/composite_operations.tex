\section{Composite Operations}
  \label{sec:compositeops}
  The ultimate goal of the ReproBLAS library is to support reproducible linear algebra functions on many widely-used architectures.

  We must account for any combination of several sources of
  nonreproducibility. These sources include arbitrary data
  permutation or layout, differing numbers of processors, and arbitrary
  reduction tree shape. The methods we will describe can be used to
  account for any or all of these sources of nonreproducibility. However, if there is only one potential source of nonreproducibility then it may be more efficient to use a method that deals specifically with that source. For example, if there is an arbitrary reduction tree shape (due to the underlying MPI implementation) but not arbitrary data layout (because this is fixed by the application) then it is probably cheaper to just to use a method
that deals with arbitrary reduction trees shapes.

  Reproducibility is a concern for several parallel architectures. Apart from the typical sequential environment, linear algebra packages can be implemented on shared-memory systems (where multiple processors operate independently and share the same memory space), distributed memory systems (where each processor has its own private memory), or even cloud computing systems (remote internet-based computing where resources are provided to users on-demand).

  The Basic Linear Algebra Subprograms (BLAS) \cite{BLAS} are widely
  used as an efficient, portable, and reliable sequential linear algebra library.
  The BLAS are divided into three categories.
  \begin{enumerate}
    \item The Level 1 BLAS (BLAS1) is a set of vector operations on
      strided arrays. Several of these operations are already
      reproducible, such as the \texttt{xscal} operation, which scales a vector by a scalar value or the \texttt{xsaxpy} operation, which adds a scaled copy of one vector to another. These operations are reproducible provided they are implemented in the same way. For example \texttt{xsaxpy} might not be reproducible if one implementation uses a fused multiply-add instruction and the other does not. Other operations, such as
      the dot product or vector norm, are not reproducible with
      respect to data permutation, and must be modified to have this quality. These operations
      are \texttt{xasum}, \texttt{xnrm2}, and 
      \texttt{xdot}.
    \item The Level 2 BLAS (BLAS2) is a set of matrix-vector operations
      including matrix-vector multiplication and a solver for $\vec{x}$ in $T\vec{x} = \vec{y}$ where $\vec{x}$ and $\vec{y}$ are vectors and $T$ is a triangular matrix. These operations can all be made reproducible, but for brevity we discuss only the representative operation \texttt{xgemv}, matrix-vector multiplication.
    \item The Level 3 BLAS (BLAS3) is a set of matrix-matrix operations
      including matrix-matrix multiplication and routines to perform $B = \alpha T^{-1}B$ where $\alpha$ is a scalar, $B$ is a matrix, and $T$ is a triangular matrix. These operations can all be made reproducible, but for brevity we discuss only the representative operation \texttt{xgemm}, matrix-matrix multiplication.
  \end{enumerate}

  The Linear Algebra PACKage (LAPACK) \cite{LAPACK}, is a set of higher-level
  routines for linear algebra, providing routines for operations like solving linear equations, linear least squares, eigenvalue problems, and singular value decomposition. LAPACK does most floating point computation using the BLAS.

  The BLAS and LAPACK have been extended to distributed-memory systems in the PBLAS \cite{PBLAS} and SCALAPACK \cite{SCALAPACK} libraries. While most BLAS and PBLAS operations can clearly be made reproducible, it is an open question as to which LAPACK and SCALAPACK routines can be made reproducible simply by using reproducible BLAS, or whether there are other sources of nonreproducibility to eliminate. This is future work.

  Out of the large design space outlined above, we describe in this paper and implement in ReproBLAS sequential versions of key operations from the BLAS, and a distributed-memory reduction operation which can be used to parallelize these sequential operations in a reproducible way. We described the BLAS1 operations \texttt{xasum} and \texttt{xdot} at the end of Section \ref{sec:primitiveops_sum} previously. The BLAS1 operation \texttt{xnrm2} is discussed in Section \ref{sec:compositeops_nrm}. The BLAS2 and BLAS3 operations \texttt{xgemv} and \texttt{xgemm} are discussed in Sections \ref{sec:compositeops_gemv} and \ref{sec:compositeops_gemm} respectively. Eventually, we intend to extend ReproBLAS to contain reproducible versions of all BLAS and PBLAS routines.

    \subsection{Reduce}
  \label{sec:compositeops_reduce}
  Perhaps the most important function to make reproducible, the parallel reduction computes the sum of $p$ floating point numbers, each number residing on a separate processor. The numbers are added pairwise in a tree. As different processors may have their number available at slightly different times, the shape of the tree may change from run to run. If a standard reduction is used, the results may differ depending on the shape of the reduction tree.

  A parallel reduction can be accomplished reproducibly by converting the floating point numbers to indexed types with the \textproc{ConvertFloatToIndexed} procedure (Algorithm \ref{alg:conv2indexed}), reducing the indexed types reproducibly using the \textproc{AddIndexedToIndexed} procedure (Algorithm \ref{alg:addindexedtoindexed}) at each node of the reduction tree, and then converting the resulting indexed sum to a floating point number with the \textproc{ConvertIndexedToFloat} procedure (Algorithm \ref{alg:conv2floatoverflow}). 

  If there are several summands on every processor, then each processor should first compute the indexed sum of the local summands using the \textproc{Sum} procedure (Algorithm \ref{alg:sum}). The $p$ resultant indexed sums can be reduced and converted to a floating point number as described above.

  It is easy to see that the previously described reduction operations produce the indexed sum of their summands by inductively applying the ``Ensure'' claim of Algorithm{alg:addindexedtoindexed}.

  An MPI data type that holds an indexed type can be created (creation is only performed once, subsequent calls return the same copy) and returned using the \texttt{idxdMPI\_DOUBLE\_COMPLEX\_INDEXED}, etc. function of \texttt{idxdMPI.h}. \texttt{idxdMPI.h} also contains the \texttt{XIXIADD} method, an MPI reduction operator that can reduce the data types in parallel.

    \subsection{Euclidean Norm}
  \label{sec:compositeops_nrm}

    \subsection{Matrix-Vector Product}
  \label{sec:compositeops_gemv}
  The matrix-vector product $y \in \R^m$ of an $m \times n$ matrix $A$ and a vector $x \in \R^n$ (denoted by $y = Ax$ and computed in the BLAS by \texttt{xgemv} \cite{BLAS}) is defined as follows (where $A_{i, j}$ is the entry in the $i^{th}$ row and $j^{th}$ column, indexing from zero for consistency):
  \[
    y_i = \sum\limits_{j = 0}^{n - 1} A_{i, j}x_j
  \]
  Before we get into disucussions of reproducibility in this situation, we must first discuss what it means for a matrix-vector product to be reproducible.
  It is clear that computation of the matrix vector product is easily distributed along different values of $i$, as rows of $A$ and copies of $x$ can be distributed among $p$ processors so that entries of $y$ may be calculated independently.
  However, in the case where $n$ is quite large compared to $m/p$, it may be necessary to parallelize calculation of the sums of each $y_i$ so that the entirety of $x$ does not need to be communicated.
  As the local sums would be computed separately and then combined, different blocking patterns or reduction tree shapes could lead to different results.
  Although the reproducible reduction discussed in Section \ref{compositeops_reduce} would gurantee reproducibility with respect to reduction tree shape, it would not guarantee reproducibility with respect to blocking pattern. This stronger guarantee can be obtained if each sum $y_i$ is calculated as an indexed sum.
  It is for this reason that we need a local version of \texttt{xgemv} that returns a vector of indexed sums, the matrix-vector product with all sums calculated using indexed summation.

  A consequence of calculating the indexed matrix-vector product using indexed sums is that it will be reproducible with respect to permutation of the columns of $A$ together with entries of $x$.
  Let $\sigma_0, ..., \sigma_{n - 1}$ be some permutation of the first $n$ nonnegative integers such that $\{\sigma_0, ..., \sigma_{n - 1}\} = \{0, ..., n - 1\}$ as sets. Then we have
  \[
    y_i = \sum\limits_{j = 0}^{n - 1} A_{i, j}x_j = \sum\limits_{j = 0}^{n - 1} A_{i, \sigma_j}x_{\sigma_j}
  \]

  More imporantly, the matrix-vector product should remain reproducible under any reduction tree shape.
  If $A = [A_{(0)}, A_{(1)}]$ where $A_{(0)}$ and $A_{(1)}$ are submatricies of size $m \times n_{(0)}$ and $m \times n_{(1)}$ and if $x = [x_{(0)}, x_{(1)}]$ where $x_{(0)}$ and $x_{(1)}$ are subvectors of size $n_{(0)}$ and $n_{(1)}$ then we have 
  \[
    Ax = A_{(0)}x_{(0)} + A_{(1)}x_{(1)}
  \]

  It is clear from Theorems \ref{thm:indexed_sum_unique} and the ``Ensure'' claim of Algorithm \ref{alg:addindexedtoindexed} that if the matrix-vector product is computed using indexed summation, the necessary properties hold.

  Actually computing the matrix-vector product using indexed summation is not difficult. In ReproBLAS, we mirror the function definition of \text{xgemv} in the BLAS as closely as possible, adding two additional parameters $\alpha$ and $\beta$ so that the entire operation performs the update $y \gets \alphaAx + \betay$ (we also add the standard parameters for transposing the matrix and selecting ordering of the matrix).

  At the core, ReproBLAS provides the function \texttt{idxdBLAS_xixgemv} in \texttt{idxdBLAS.h}. This version of the function adds to the vector of indexed sums $y$ the previously mentioned indexed matrix vector product of the floating point $A$ and $x$, where $x$ is first scaled by $\alpha$ as is done in the reference BLAS implementation \cite{netlib}. It is important to notice that the parameter $\beta$ is excluded, as we do not yet know how to scale indexed types by different values while maintaining reproducibility.
  It is of note that because the reproducible dot product is so compute-heavy, we can get good performance implementing the matrix-vector product using the \texttt{idxdBLAS_diddot} routine at the core. We must use a rectangular blocking strategy to ensure good caching behavior, and because we are making calls to the dot product routine, it is sometimes (depending on the transpose and row-major vs. column-major parameters) necessary to transpose each block of $A$ before computing the dot product. These considerations ensure that the dot product can compute quickly on sequential sections of cached data.

  Built on top of \texttt{idxdBLAS_xixgemv} is the routine \texttt{reproBLAS_rxgemv} in \texttt{reproBLAS.h}, which takes as input floating point $A$, $x$, $y$, $\alpha$, and $\beta$. $y$ is scaled by $\beta$, then converted to a vector of indexed types. The matrix-vector product is computed using indexed summation with a user specified number of accumulators, and the output is then converted back to floating point and returned. The routine \texttt{reproBLAS_xgemv} uses the default number of accumulators.

    \subsection{Matrix-Matrix Product}
  \label{sec:compositeops_gemm}
  The $m \times n$ matrix-matrix product $C$ of an $m \times k$ matrix $A$ and a $k \times n$ matrix $B$ where $\alpha, \beta \in \R$ (referred to as $C = AB$ and computed in the BLAS by \texttt{xgemm}) is defined as follows (where $A_{i, j}$ is the entry in the $i^{th}$ row and $j^{th}$ column of $A$, indexing from zero for consistency):
  \[
    C_{i, j} = \beta C_{i, j} + \sum\limits_{l = 0}^{k - 1} \alpha A_{i, l}B_{l, j}
  \]
  The matrix-matrix product is a similar construction to the matrix-vector product, and discussion of reproducibility proceeds similarly.
  Computation of the matrix-matrix product can be distributed in a myriad of ways. SUMMA (Scalable Universal Matrix Multiply Algorithm) \cite{SUMMA} is used in the PBLAS, the most popular parallel BLAS implementation. In SUMMA and almost all other parallel matrix-matrix algorithms, the computation is broken up into rectangular blocks of A, B, and C. Although SUMMA has a static reduction tree (assuming a fixed number of processors), this cannot be assumed for all parallel matrix multiply algorithms, so it may be necessary to use a reproducible reduction step (discussed in Section \ref{sec:compositeops_reduce}).
  Even if the reduction tree is reproducible for a particular blocking pattern, the blocking pattern may not be reproducible (for instance, if more or fewer processors are used). Therefore, we must compute the matrix-matrix product using indexed summation.

  It is for this reason that we need a local version of \texttt{xgemm} that returns a matrix of indexed sums, the matrix-matrix product with all sums calculated using indexed summation.

  A consequence of calculating the indexed matrix-matrix product using indexed sums is that it will be reproducible with respect to permutation of the columns of $A$ together with rows of $B$.
  Let $\sigma_0, ..., \sigma_{k - 1}$ be some permutation of the first $k$ nonnegative integers such that $\{\sigma_0, ..., \sigma_{k - 1}\} = \{0, ..., n - 1\}$ as sets. Then we have
  \begin{equation}
    C_{i, j} = \sum\limits_{l = 0}^{k - 1} A_{i, l}B_{l, j} = \sum\limits_{l = 0}^{k - 1} A_{i, \sigma_l}B_{\sigma_l, j}
    \label{eq:gemmpermute}
  \end{equation}

  More importantly, the matrix-vector product should remain reproducible under any reduction tree shape.
  If $A = [A_{(0)}, A_{(1)}]$ where $A_{(0)}$ and $A_{(1)}$ are submatrices of size $m \times k_{(0)}$ and $m \times k_{(1)}$ and if $B = \left[\begin{array}{c}B_{(0)}\\ B_{(1)}\end{array}\right]$ where $B_{(0)}$ and $B_{(1)}$ are submatrices of size $k_{(0)} \times n$ and $k_{(1)} \times n$ then we have 
  \begin{equation}
    AB = A_{(0)}B_{(0)} + A_{(1)}B_{(1)}
    \label{eq:gemmblock}
  \end{equation}

  It is clear from Theorems \ref{thm:indexed_sum_unique} and the ``Ensure'' claim of Algorithm \ref{alg:addindexedtoindexed} that if the matrix-matrix product is computed using indexed summation, the result is reproducible.

  Like the matrix-vector product, we can compute the matrix-matrix product using indexed summation with some function calls to \texttt{idxdBLAS\_xixdot}. In ReproBLAS, we mirror the function definition of \texttt{xgemm} in the BLAS as closely as possible, adding two additional parameters $\alpha$ and $\beta$ so that the entire operation performs the update $C \gets \alpha AB + \beta C$ (we also add the standard parameters for transposing the matrices and selecting row-major or column-major ordering of all matrices).

  At the core, ReproBLAS provides the function \texttt{idxdBLAS\_xixgemm} in \texttt{idxdBLAS.h} (see Section \ref{sec:reproBLAS} for details). This version of the function adds to the matrix of indexed sums $C$ the previously mentioned indexed matrix-matrix product of the floating point $A$ and $B$, where $x$ is first scaled by $\alpha$ as is done in the reference BLAS implementation. 
To be clear, \texttt{idxdBLAS\_xixgemm} assumes that $C$ is a matrix of indexed types and that all other inputs are floating point. A version (\texttt{reproBLAS\_rxgemm}) of the matrix matrix product routine that assumes $C$ to be a floating point matrix is discussed later. 
Again the parameter $\beta$ is excluded when $C$ is composed of indexed types (but not when $C$ is composed of floating point numbers which will be discussed below), as we do not yet know how to scale indexed types by different values (other than $0$, $1$, or $-1$) while maintaining reproducibility.

  Again because the reproducible dot product is so compute-heavy, we can get good performance implementing the matrix-matrix product using the \texttt{idxdBLAS\_diddot} routine at the core. The blocking strategy is more complicated this time, however, as computation can proceed under several loop orderings. Because the matrices $A$ and $B$ are composed of single floating point entries and $C$ is composed of the much larger indexed types (each indexed type usually contains at least $6 = 2 * K$ of its constituent floating-point types), we chose to completely compute blocks of $C$ by iterating over the matrices $A$ and $B$. This strategy avoids having to perform multiple iterations over the matrix $C$ composed of much larger data types. 
  Again, to keep the dot product running smoothly we first transpose blocks of $A$ and/or $B$ (depending on the transpose and row-major or column-major ordering options) when it is necessary to obtain contiguous sequences of cached data.

  Built on top of \texttt{idxdBLAS\_xixgemm} is the routine \texttt{reproBLAS\_rxgemm} in \texttt{reproBLAS.h} (see Section \ref{sec:reproBLAS} for details), which takes as input floating point $A$, $B$, $C$, $\alpha$, and $\beta$. $C$ is scaled by $\beta$, then converted to a vector of indexed types. The matrix-matrix product is computed using indexed summation with a user specified number of accumulators, and the output is then converted back to floating point and returned. The routine \texttt{reproBLAS\_xgemm} uses the default number of accumulators.

