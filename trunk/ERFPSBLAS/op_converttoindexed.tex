\subsection{Convert Float to Indexed}
    \label{sec:primitiveops_conv2indexed}
    Converting a floating point number to an indexed type should produce, for
    transparency and reproducibility, the indexed sum of the single floating
    point number.
    The procedure is very simply summarized by Algorithm \ref{alg:conv2indexed}, and is available in ReproBLAS as \texttt{idxd\_xixconv} in \texttt{idxd.h} (see Section \ref{sec:reproBLAS} for details).

    \begin{samepage}
    \begin{alg}
      Convert floating point $x$ to a $K$-fold indexed type $Y$.
      \begin{algorithmic}[1]
        \Function{ConvertFloatToIndexed}{K, x, Y}
          \State $Y = 0$
          \State \Call{Update}{K, x, Y}
          \State \Call{Deposit}{K, x, Y} \label{alg:conv2indexed:deposit}
          \State \Call{Renorm}{K, Y}
        \EndFunction
        \Ensure
          $Y$ is the indexed sum of $x$. $Y=0$ if $x=0$.
          When $x \neq 0$, the index $I$ of $Y$ is such that $2^{b_I} > |x| \geq 2^{a_I}$
          if $|x| \geq 2^{a_{i_max}}$, otherwise $I = i_{max}$ where $i_{max}$ is defined as \eqref{eq:imax}.
          ${Y_k}_P \in [1.5  \epsilon^{-1} 2^{a_{I + k}}, 1.75  \epsilon^{-1} 2^{a_{I + k}})$
      \end{algorithmic}
      \label{alg:conv2indexed}
    \end{alg}
    \end{samepage}

    We must renormalize the indexed type because $d(x, I + k)$ could be
    negative, meaning ${Y_k}_P < 1.5  \epsilon^{-1} 2^{a_{I + k}}$ after
    execution of line \ref{alg:conv2indexed:deposit}.
