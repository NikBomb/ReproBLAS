\subsection{Renormalize}
    \label{sec:primitiveops_renormalize}
    When depositing values into a $K$-fold indexed type $Y$ of index $I$,
    Algorithms \ref{alg:depositrestricted} and \ref{alg:deposit}
    assume that 
    \(
      {Y_k}_P \in (\epsilon^{-1} 2^{a_{I + k}}, 2 \epsilon^{-1} 2^{a_{I + k}})
    \) throughout the routine.
    To enforce this condition, the indexed type must be renormalized at least
    every $2^{p-W-2}$ deposit operations, as will be shown in Theorem \ref{thm:renormfreq}.
    The renormalization procedure is shown in Algorithm \ref{alg:renorm},
    which works for all indices $0 \leq I \leq i_{max}$.

    \begin{samepage}
    \begin{alg}
      For a $K$-fold indexed type $Y$ of index $I$, assuming ${Y_k}_P \in [1.25  \epsilon^{-1} 2^{a_{I + k}}, 2  \epsilon^{-1} 2^{a_{I + k}})$, renormalize $Y$ such that ${Y_k}_P \in [1.5  \epsilon^{-1} 2^{a_{I + k}}, 1.75  \epsilon^{-1} 2^{a_{I + k}})$.
      \begin{algorithmic}[1]
        \Require
          \Statex ${Y_k}_P \in [1.25  \epsilon^{-1} 2^{a_{I + k}}, 2  \epsilon^{-1} 2^{a_{I + k}})$
        \Function{Renorm}{K, Y}
          \For{$k = 0 \To K - 1$}
            \If{${Y_k}_P < 1.5 \cdot \ufp({Y_k}_P)$}
              \State ${Y_k}_P = {Y_k}_P + 0.25 \cdot \ufp({Y_k}_P)$
              \State ${Y_k}_C = {Y_k}_C - 1$
            \EndIf
            \If{${Y_k}_P \geq 1.75 \cdot \ufp({Y_k}_P)$}
              \State ${Y_k}_P = {Y_k}_P - 0.25 \cdot \ufp({Y_k}_P)$
              \State ${Y_k}_C = {Y_k}_C + 1$
            \EndIf
          \EndFor
        \EndFunction
        \Ensure
          \Statex ${Y_k}_P \in [1.5  \epsilon^{-1} 2^{a_{I + k}}, 1.75  \epsilon^{-1} 2^{a_{I + k}})$.
          \Statex The values $\mathcal{Y}_k$ are unchanged. Recall that by \eqref{eq:acc},
          \begin{equation*}
            \mathcal{Y}_k = {\mathcal{Y}_k}_P + {\mathcal{Y}_k}_C = ({Y_k}_P - 1.5 \epsilon^{-1} 2^{a_{I + k}}) + (0.25\epsilon^{-1}2^{a_{I + k}}){Y_k}_C
          \end{equation*}
      \end{algorithmic}
      \label{alg:renorm}
    \end{alg}
    \end{samepage}
    The renormalization operation is described in the ``Carry-bit Propagation''
    Section (lines 21 to 32) of Algorithm $6$ in \cite{repsum}, although it has
    been slightly modified so as not to include an extraneous case. Note that
    the above renormalization algorithm can be modified to tolerate
    indexed types with exceptional values by doing nothing when such types are
    encountered (indexed types with exceptional values do not need
    renormalization).
    In total, Algorithm~\ref{alg:renorm} costs $3K$ FLOPs with a maximum of
    $2K$ conditional branches.

    To show the reasoning behind the assumptions in Algorithm \ref{alg:renorm},
    we prove Theorem \ref{thm:renormfreq}.

      \begin{samepage}
    \begin{thm}
      Assume $x_0, x_1, ... x_{n - 1} \in \F$ are successively deposited (using Algorithm \ref{alg:deposit}) in a $K$-fold indexed type $Y$ of index $I$ where $\max|x_j| < 2^{b_I}$. If $Y$ initially satisfies ${Y_k}_P \in [1.5  \epsilon^{-1} 2^{a_{I + k}}, 1.75  \epsilon^{-1} 2^{a_{I + k}})$ and $n \leq 2^{p - W - 2}$, then after all of the deposits, ${Y_k}_P \in [1.25  \epsilon^{-1} 2^{a_{I + k}}, 2  \epsilon^{-1} 2^{a_{I + k}})$.
      \label{thm:renormfreq}
    \end{thm}
    \end{samepage}

    \begin{proof}
    First, note that $|d(x_j, I + k)| \leq 2^{b_{I + k}}$ by Theorem
    \ref{thm:dbound}, where $d(x_j, I + k)$ is the amount added to ${Y_k}_P$ on
    iteration $k$.

    By Theorem \ref{thm:ddeposit}, the deposit operation extracts and adds the slices of $x_j$ exactly (assuming ${Y_k}_P \in (\epsilon^{-1} 2^{a_{I + k}}, 2  \epsilon^{-1} 2^{a_{I + k}})$ at each step, which will be shown),

    \begin{equation*}
    \bigl|\sum \limits_{j = 0}^{n - 1} d(x_j, I + k)\bigr| \leq n  2^{b_{I + k}} = n  2^{W}  2^{a_{I + k}}
    \end{equation*}

    If $n \leq 2^{p - W - 2}$, then after the $n^{th}$ deposit

    \begin{align*}
    {Y_k}_P &\in \bigl[(1.5  \epsilon^{-1} - n  2^W) 2^{a_{I + k}}, (1.75  \epsilon^{-1} + n  2^W) 2^{a_{I + k}}\bigr) \\
    &\in [1.25  \epsilon^{-1} 2^{a_{I + k}}, 2  \epsilon^{-1} 2^{a_{I + k}})
    \end{align*}
    \end{proof}

    If an indexed type $Y$ initially satisfies ${Y_k}_P \in [1.5  \epsilon^{-1}
    2^{a_{I + k}}, 1.75  \epsilon^{-1} 2^{a_{I + k}})$
    (such a condition is satisfied upon initialization of a new accumulator of $Y$
    during the updating process as will be explained in Section~\ref{sec:primitiveops_update})
    and we deposit at most
    $2^{p-W-2}$ floating point numbers into it, then Theorem \ref{thm:renormfreq}
    shows that after all of the deposits, ${Y_k}_P \in [1.25  \epsilon^{-1}
    2^{a_{I + k}}, 2  \epsilon^{-1} 2^{a_{I + k}})$. Therefore, after another
    renormalization, the primary fields would once again satisfy ${Y_k}_P \in
    [1.5  \epsilon^{-1} 2^{a_{I + k}}, 1.75  \epsilon^{-1} 2^{a_{I + k}})$.
    The limit on the number of floating point inputs that can be accumulated
    without executing a renormalization operation ($2^{p-W-2}$)
    also requires that $p-W-2 > 0$, or
    \begin{equation}
        W < p - 2.
    \end{equation}

    As ${Y_k}_C$ must be able to record additions of absolute value 1 without
    error, ${Y_k}_C$ must stay in the range $[-\epsilon^{-1}, \epsilon^{-1}]$.
    As each renormalization results in addition not in excess absolute value of
    1 to ${Y_k}_C$, a maximum of $\epsilon^{-1}$ renormalizations may be
    performed, meaning that an indexed type is capable of representing the sum
    of at least
    \begin{equation}
      \epsilon^{-1} 2^{p-W-2} = 2^{2  p - W - 2}
      \label{eq:totalfreq}
    \end{equation}
    floating point numbers. These maxima are $2^{64}$ in double and $2^{33}$ in single
    precision using the values in Table~\ref{fig:bins}.

    Note that this value of maximum number of additions is slightly bigger
    than that of \cite{repsum} since we exclude the extraneous case which caused
    the increment of the carry field to at most 2 in absolute value per each renormalization.
    These bounds also exceed the largest integer that can be represented
    in the same-sized integer format, which helps justify the choice of $W$.
