\subsection{Renormalize}
    \label{sec:primitiveops_renormalize}
    When depositing values into a $K$-fold indexed type $Y$ of index $I$,
    Algorithms \ref{alg:depositrestricted} and \ref{alg:deposit}
    assume that 
    \[
      {Y_k}_P \in \begin{cases}[1.5  \epsilon^{-1} 2^{a_{I + k}}, 1.75  \epsilon^{-1} 2^{a_{I + k}}) \text{ if } I + k > 0 \\ [1.5 \cdot 2^{e_{\max}}, 1.75 \cdot 2^{e_{\max}}) \text{ if } I + k = 0\end{cases}
    \] 
    throughout the routine.
    To enforce this condition, the indexed type must be \textbf{renormalized} at least
    every $2^{p-W-2}$ deposit operations, as will be shown in Theorem \ref{thm:renormfreq}.
    The renormalization procedure is shown in Algorithm \ref{alg:renorm},
    which works for all indices $0 \leq I \leq i_{max}$. Algorithm \ref{alg:renorm} is available in ReproBLAS as \texttt{idxd\_xirenorm} in \texttt{idxd.h} (see Section \ref{sec:reproBLAS} for details).

    \begin{samepage}
    \begin{alg}
      Renormalize a $K$-fold indexed type $Y$ of index $I$.
      \begin{algorithmic}[1]
        \Require
        \Statex ${Y_k}_P \in \begin{cases}[1.25  \epsilon^{-1} 2^{a_{I + k}}, 2  \epsilon^{-1} 2^{a_{I + k}}) \text{ if } I + k > 0 \\ [1.25 \cdot 2^{e_{\max}}, 2 \cdot 2^{e_{\max}}) \text{ if } I + k = 0\end{cases} $
        \Function{Renorm}{K, Y}
          \For{$k = 0 \To K - 1$}
            \If{${Y_k}_P < 1.5 \cdot \ufp({Y_k}_P)$}
              \State ${Y_k}_P = {Y_k}_P + 0.25 \cdot \ufp({Y_k}_P)$
              \State ${Y_k}_C = {Y_k}_C - 1$
            \EndIf
            \If{${Y_k}_P \geq 1.75 \cdot \ufp({Y_k}_P)$}
              \State ${Y_k}_P = {Y_k}_P - 0.25 \cdot \ufp({Y_k}_P)$
              \State ${Y_k}_C = {Y_k}_C + 1$
            \EndIf
          \EndFor
        \EndFunction
        \Ensure
          \Statex ${Y_k}_P \in \begin{cases}[1.5  \epsilon^{-1} 2^{a_{I + k}}, 1.75  \epsilon^{-1} 2^{a_{I + k}}) \text{ if } I + k > 0 \\ [1.5 \cdot 2^{e_{\max}}, 1.75 \cdot 2^{e_{\max}}) \text{ if } I + k = 0\end{cases} $
          \Statex The values $\mathcal{Y}_k$ are unchanged. Recall that by \eqref{eq:acc}, if $I + k > 0$,
          \begin{equation*}
            \mathcal{Y}_k = {\mathcal{Y}_k}_P + {\mathcal{Y}_k}_C = ({Y_k}_P - 1.5 \epsilon^{-1} 2^{a_{I + k}}) + (0.25\epsilon^{-1}2^{a_{I + k}}){Y_k}_C
          \end{equation*}
      \end{algorithmic}
      \label{alg:renorm}
    \end{alg}
    \end{samepage}
    The renormalization operation is described in the ``Carry-bit Propagation''
    Section (lines 21 to 32) of Algorithm $6$ in \cite{repsum}, although it has
    been slightly modified so as not to include an extraneous case. Indexed 
    types with exceptional values do not need renormalization. 
    Algorithm \ref{alg:renorm} can be modified to handle indexed 
    types with exceptional values by doing nothing when such types are
    encountered (depending on how $\ufp()$ behaves when given exceptional values, Algorithm \ref{alg:renorm} could change $\pm\texttt{Inf}$ to \texttt{NaN}).
    In total, Algorithm~\ref{alg:renorm} costs $3K$ FLOPS with a maximum of
    $2K$ conditional branches.

    To show the reasoning behind the assumptions in Algorithm \ref{alg:renorm},
    we prove Theorem \ref{thm:renormfreq}.

      \begin{samepage}
    \begin{thm}
      Assume $x_0, x_1, ... x_{n - 1} \in \F$ are successively deposited (using Algorithm \ref{alg:deposit}) in a $K$-fold indexed type $Y$ of index $I$ where $\max|x_j| < 2^{b_I}$. If $Y$ initially satisfies
       \[
      {Y_k}_P \in \begin{cases}[1.5  \epsilon^{-1} 2^{a_{I + k}}, 1.75  \epsilon^{-1} 2^{a_{I + k}}) \text{ if } I + k > 0 \\ [1.5 \cdot 2^{e_{\max}}, 1.75 \cdot 2^{e_{\max}}) \text{ if } I + k = 0\end{cases}
      \]
       and $n \leq 2^{p - W - 2}$, then after all of the deposits, 
       \[
      {Y_k}_P \in \begin{cases}[1.25  \epsilon^{-1} 2^{a_{I + k}}, 2  \epsilon^{-1} 2^{a_{I + k}}) \text{ if } I + k > 0 \\ [1.25 \cdot 2^{e_{\max}}, 2 \cdot 2^{e_{\max}}) \text{ if } I + k = 0\end{cases}
      \]
      \label{thm:renormfreq}
    \end{thm}
    \end{samepage}

    \begin{proof}
    As the proof when $I + k = 0$ is almost identical to the case where $I + k > 0$, we consider here only the case that $I + k > 0$.
    First, note that $|d(x_j, I + k)| \leq 2^{b_{I + k}}$ by Theorem
    \ref{thm:dbound}, where $d(x_j, I + k)$ is the amount added to ${Y_k}_P$ on
    iteration $k$.

    By Theorem \ref{thm:ddeposit}, \textproc{Deposit} (Algorithm \ref{alg:deposit}) extracts and adds the slices of $x_j$ exactly (assuming ${Y_k}_P \in (\epsilon^{-1} 2^{a_{I + k}}, 2  \epsilon^{-1} 2^{a_{I + k}})$ at each step, which will be shown),

    \begin{equation*}
    \bigl|\sum \limits_{j = 0}^{n - 1} d(x_j, I + k)\bigr| \leq n  2^{b_{I + k}} = n  2^{W}  2^{a_{I + k}}
    \end{equation*}

    If $n \leq 2^{p - W - 2}$, then after the $n^{th}$ deposit

    \begin{align*}
    {Y_k}_P &\in \bigl[(1.5  \epsilon^{-1} - n  2^W) 2^{a_{I + k}}, (1.75  \epsilon^{-1} + n  2^W) 2^{a_{I + k}}\bigr) \\
    &\in [1.25  \epsilon^{-1} 2^{a_{I + k}}, 2  \epsilon^{-1} 2^{a_{I + k}})
    \end{align*}
    \end{proof}

    If an indexed type $Y$ initially satisfies ${Y_k}_P \in [1.5  \epsilon^{-1}
    2^{a_{I + k}}, 1.75  \epsilon^{-1} 2^{a_{I + k}})$
    (such a condition is satisfied upon initialization of a new accumulator of $Y$
    during the updating process as will be explained in Section~\ref{sec:primitiveops_update})
    and we deposit at most
    $2^{p-W-2}$ floating point numbers into it, then Theorem \ref{thm:renormfreq}
    shows that after all of the deposits, ${Y_k}_P \in [1.25  \epsilon^{-1}
    2^{a_{I + k}}, 2  \epsilon^{-1} 2^{a_{I + k}})$. Therefore, after another
    renormalization, the primary fields would once again satisfy ${Y_k}_P \in
    [1.5  \epsilon^{-1} 2^{a_{I + k}}, 1.75  \epsilon^{-1} 2^{a_{I + k}})$.
    The limit on the number of floating point inputs that can be accumulated
    without executing a renormalization operation ($2^{p-W-2}$)
    also requires that $p-W-2 > 0$, or
    \begin{equation}
        W < p - 2.
    \end{equation}

    As ${Y_k}_C$ must be able to record additions of absolute value 1 without
    error, ${Y_k}_C$ must stay in the range $[-\epsilon^{-1}, \epsilon^{-1}]$.
    As each renormalization results in addition not in excess absolute value of
    1 to ${Y_k}_C$, a maximum of $\epsilon^{-1} - 1$ renormalizations may be
    performed, meaning that an indexed type is capable of representing the sum
    of at least
    \begin{equation}
      (\epsilon^{-1} - 1) 2^{p-W-2} \approx 2^{2  p - W - 2}
      \label{eq:totalfreq}
    \end{equation}
    floating point numbers. The value of $(\epsilon^{-1} - 1)2^{p-W-2}$ is approximately $2^{64}$ in double and $2^{33}$ in single
    precision using the values in Table~\ref{tbl:bins}.

    Note that this value of maximum number of additions is slightly bigger
    than that of \cite{repsum} since we exclude the extraneous case which caused
    the increment of the carry field to at most 2 in absolute value per each renormalization.
    These bounds also exceed the largest integer that can be represented
    in the same-sized integer format, which helps justify the choice of $W$.
