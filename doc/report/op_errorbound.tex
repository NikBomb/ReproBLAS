\subsection{Error Bound}
    \label{sec:primitiveops_error}

    We first state and prove Theorem \ref{thm:mysortsum}, as it is critical in
    the error analysis of Algorithms \ref{alg:conv2float} and \ref{alg:conv2floatoverflow}. It should be noted that Theorem
    \ref{thm:mysortsum} is similar to that of Theorem 1 from \cite{sortsum},
    but requires less intermediate precision by exploiting additional structure
    of the input data.

    As discussed in Section \ref{sec:primitiveops_conv2float}, to convert an
    indexed type to a floating point number, one must evaluate equation
    \ref{eq:indexedvalue} accurately and without unnecessary overflow. This amounts
    to summing the fields of the indexed type.
    It is possible that future implementers may make modifications to the
    indexed type (adding multiple carry fields, changing the binning scheme,
    etc.) such that the summation of its fields cannot be reordered to satisfy
    the assumptions of Theorem \ref{thm:mysortsum}. In such an event,
    $\cite{sortsum}$ provides more general ways to sum the fields while still
    maintaining accuracy.
      \begin{samepage}
    \begin{thm}
      We are given $n$ floating point numbers $f_0, \ldots, f_{n - 1}$ for which there
      exist (possibly unnormalized) floating point numbers $f'_0, \ldots, f'_{n -1}$
      of the same precision such that
      \begin{enumerate}
        \item $f_j = f'_j$ for all $j \in \{0, ..., n - 1\}$
        \item $\exp(f'_0) > ... > \exp(f'_{n - 1})$
        \item $\exp(f'_j) \geq \exp(f'_{j + 2}) + \lceil\frac{p + 1}{2}\rceil$ for all $j \in \{0, ..., n - 3\}$
      \end{enumerate}
      \label{thm:mysortsum}
      Let $S_0 = \overline{S_0} = f_0$, $S_j = S_{j - 1} + f_j$,
      and $\overline{S_j} = \fl(\overline{S_{j - 1}} + f_j)$ (assuming rounding to nearest)
      so that $S_{n - 1} = \sum \limits_{j = 0}^{n - 1} f_j$.
      Then in the absence of overflow and underflow we have
      \begin{equation*}
        \left|S_{n - 1} - \overline{S_{n - 1}}\right| < \frac{7\epsilon}{1 - 6\sqrt\epsilon}|S_{n - 1}| \approx 7 \epsilon |S_{n - 1}|
      \end{equation*}
    \end{thm}
    \end{samepage}

    \begin{proof}

      Throughout the proof, let $f_j = 0$ if $j > n - 1$ so that $S_{\infty} = S_{n - 1}$ and $\overline{S_{\infty}} = \overline{S_{n - 1}}$.

      Let $m$ be the location of the first error such that $S_{m - 1} = \overline{S_{m - 1}}$ and $S_{m} \neq \overline{S_{m}}$.

      If no such $m$ exists then the computed sum is exact ($S_{n - 1} = \overline{S_{n - 1}}$) and we are done.

      If such an $m$ exists, then because $\exp(f_0') > ... > \exp(f_m')$, $f_0, ..., f_m \in \ulp(f_m')\Z$. Thus, $S_m \in \ulp(f_m')\Z$.

      We now show $|S_m| > 2 \cdot 2^{\exp(f_m')}$. Assume for contradiction that $|S_m| \leq 2 \cdot 2^{\exp(f_m')}$. Because $S_m \in \ulp(f_m')\Z$, this would imply that $S_m$ is representable as a floating point number, a contradiction as $\overline{S_m} \neq S_m$. Therefore, we have
      \begin{equation}
        |S_m| > 2 \cdot 2^{\exp(f_m')}
        \label{eq:smbound}
      \end{equation}

      Because $\exp(f_m') > \exp(f_{m + 1}')$,
      \begin{equation}
        |f_{m + 1}| < 2\cdot2^{\exp(f_m' - 1)} = 2^{\exp(f_m')}
        \label{eq:smpbound}
      \end{equation}

      Because $\exp(f_m') \geq \exp(f_{m + 2}') + \lceil\frac{p + 1}{2}\rceil$ and $\exp(f_0') > ... > \exp(f_{n - 1}')$,
      \begin{align}
        \bigl|\sum \limits_{j = m + 2}^{n - 1} f_j\bigr| &\leq \sum \limits_{j = m + 2}^{n - 1} |f_j| < \sum \limits_{j = m + 2}^{n - 1} 2 \cdot 2^{\exp(f_j')} \leq \sum \limits_{j = m + 2}^{n - 1} 2 \cdot 2^{\exp(f_m') - \left\lceil\frac{p + 1}{2}\right\rceil - (m + 2 - j)} \nonumber \\
        &< \sum \limits_{j = 0}^{\infty} \left(2 \sqrt{\epsilon}\right)2^{\exp(f_m') - j} = \left(4\sqrt\epsilon\right)2^{\exp(f_m')}
        \label{eq:smppbound}
      \end{align}

      We can combine  \eqref{eq:smpbound} and \eqref{eq:smppbound} to obtain
      \begin{equation}
        \bigl|\sum\limits_{j = m + 1}^{n - 1} f_j\bigr| \leq \sum\limits_{j = m + 1}^{n - 1} |f_j| < 2^{\exp{f_m'}} + \left(4 \sqrt{\epsilon}\right) 2^{\exp(f_m')} = \left(1 + 4 \sqrt\epsilon \right)2^{\exp(f_m')}
        \label{eq:smsbound}
      \end{equation}

      By  \eqref{eq:smbound} and \eqref{eq:smsbound},
      \begin{align}
        |S_{n-1}| & = \bigl|\sum\limits_{j = 0}^{n - 1} f_j\bigr| \geq \bigl|\sum\limits_{j = 0}^{m} f_j\bigr| - \bigl|\sum\limits_{j = m + 1}^{n - 1} f_j\bigr| = |S_m| - \bigl|\sum\limits_{j = m + 1}^{n - 1} f_j\bigr| \nonumber \\
        & \geq 2 \cdot 2^{\exp(f_{m}')} - \left(1 + 4 \sqrt\epsilon\right) 2^{\exp(f_m')} = \left(1 - 4 \sqrt\epsilon\right) 2^{\exp(f_m')}
        \label{eq:sbound}
      \end{align}

      By  \eqref{eq:sbound} and \eqref{eq:smppbound},
      \begin{equation}
        \bigl|\sum \limits_{j = m + 2}^{n - 1} f_j\bigr| < \left(4 \sqrt{\epsilon}\right) 2^{\exp(f_m')} \leq \frac{4 \sqrt\epsilon}{1 - 4  \sqrt\epsilon}\bigl|\sum\limits_{j = 0}^{n - 1}f_j\bigr|
        \label{eq:smpprelsbound}
      \end{equation}

      By  \eqref{eq:sbound} and \eqref{eq:smsbound},
      \begin{equation}
        \bigl|\sum\limits_{j = m + 1}^{n - 1}f_j\bigr| \leq \sum\limits_{j = m + 1}^{n - 1}|f_j| \leq \left(1 + 4  \sqrt\epsilon\right)2^{\exp(f_m')}\leq \frac{1 + 4  \sqrt\epsilon}{1 - 4  \sqrt\epsilon}\bigl|\sum\limits_{j = 0}^{n - 1}f_j\bigr|
        \label{eq:smsrelsbound}
      \end{equation}

      And by \eqref{eq:sbound} and \eqref{eq:smsrelsbound},
      \begin{equation}
        |S_m| \leq \bigl|\sum\limits_{j = 0}^{n - 1}f_j\bigr| + \bigl|\sum\limits_{j = m + 1}^{n - 1} f_j\bigr|
            \leq \left(1 + \frac{1 + 4\sqrt\epsilon}{1 - 4\sqrt\epsilon}\right)\bigl|\sum_{j = 0}^{n - 1}f_j\bigr|
            = \frac{2}{1 - 4  \sqrt\epsilon}\bigl|\sum\limits_{j = 0}^{n - 1}f_j\bigr|
        \label{eq:smrelsbound}
      \end{equation}

      By definition, $\overline{S_{m+4}}$ is the computed sum of
      $\overline{S_m}$, $f_{m+1}, \ldots, f_{m+4}$ using the standard recursive summation technique.
      According to \cite[Equation 1.2, 2.4]{higham}
      \begin{align*}
          \bigl|\overline{S_m} + \sum_{j=m+1}^{m+4}f_j - \overline{S_{m+4}}\bigr|
          & \leq \frac{4\epsilon}{1-4\epsilon} \left|\overline{S_m} + f_{m+1}\right| + \frac{3\epsilon}{1-3\epsilon} \sum_{j=m+2}^{m+4}|f_j| \\
          & \leq \frac{4\epsilon}{1-4\epsilon} \bigl(\left|\overline{S_m} - S_m\right| + |S_m + f_{m+1}|\bigr)
              + \frac{3\epsilon}{1-3\epsilon} \sum_{j=m+2}^{n-1}|f_j|.
      \end{align*}
      Since $S_{n-1} = S_m + f_{m+1} + \sum_{j=m+2}^{n-1} f_j$, we have
      \begin{equation*}
          |S_m + f_{m+1}|
          = \bigl|S_{n-1} - \sum_{j=m+2}^{n-1}f_j\bigr|
          \leq |S_{n-1}| + \sum_{j=m+2}^{n-1} |f_j|
      \end{equation*}
      Therefore
      \begin{equation*}
          \bigl|\overline{S_m} + \sum_{j=m+1}^{m+4}f_j - \overline{S_{m+4}}\bigr|
          \leq \frac{4\epsilon}{1-4\epsilon} \left|S_m - \overline{S_m}\right|
          + \frac{4\epsilon}{1-4\epsilon} |S_{n-1}|
          + \frac{7\epsilon}{1-4\epsilon} \sum_{j=m+2}^{n-1}|f_j|.
      \end{equation*}
      Using the triangle inequality we have
      \begin{align*}
      \left|S_{m+4} - \overline{S_{m+4}}\right|
          & = \bigl|S_m + \sum_{j=m+1}^{m+4}f_j - \overline{S_{m+4}}\bigr|
          \leq \left|S_m - \overline{S_m} \right| + \bigl|\overline{S_m} + \sum_{j=m+1}^{m+4}f_j - \overline{S_{m+4}} \bigr| \\
          & \leq \left(1 + \frac{4\epsilon}{1-4\epsilon}\right) \left|S_m - \overline{S_m}\right| + \frac{4\epsilon}{1-4\epsilon} |S_{n-1}|
                  + \frac{7\epsilon}{1-4\epsilon} \sum_{j=m+2}^{n-1}|f_j| \\
          & \leq \frac{1}{1-4\epsilon} \epsilon |S_m| + \frac{4\epsilon}{1-4\epsilon} |S_{n-1}|
                  + \frac{7\epsilon}{1-4\epsilon} \sum_{j=m+2}^{n-1}|f_j| \\
          & \leq \frac{\epsilon}{1-4\epsilon} \left(|S_m| + 4 |S_{n-1}|
                  + 7 \sum_{j=m+2}^{n-1}|f_j|\right).
      \end{align*}
      and by \eqref{eq:smrelsbound} and \eqref{eq:smpprelsbound},
      \begin{align}
      \left|S_{m+4} - \overline{S_{m+4}}\right|
          & \leq \frac{\epsilon}{1-4\epsilon}
              \left(
                  \frac{2}{1-4\sqrt{\epsilon}} |S_{n-1}|
                  + 4 |S_{n-1}|
                  + 7 \frac{4\sqrt{\epsilon}}{1-4\sqrt{\epsilon}} |S_{n-1}|
              \right) \nonumber \\
          & \leq \frac{\epsilon}{1-4\epsilon} \left(\frac{6+12\sqrt{\epsilon}}{1-4\sqrt{\epsilon}} |S_{n-1}|\right)
              = \frac{6\epsilon }{(1-2\sqrt{\epsilon})(1-4\sqrt{\epsilon})} |S_{n-1}| \nonumber \\
          & < \frac{6\epsilon}{1-6\sqrt{\epsilon}} |S_{n-1}|
          \label{eq:smfiveerror}
      \end{align}

      Notice that
      \begin{equation*}
        \exp(f_m') \geq \exp(f_{m + 2}') + \left\lceil\frac{p+ 1}{2}\right\rceil \geq \exp(f_{m + 4}') + 2  \left\lceil\frac{p + 1}{2}\right\rceil > \exp(f_{m + 5}')+ 2  \left\lceil\frac{p+ 1}{2}\right\rceil
      \end{equation*}
      Therefore,
      \begin{equation}
        \exp(f_m') \geq \exp(f_{m + 5}') + p + 2
        \label{eq:fmfiveexp}
      \end{equation}

      Because $\exp(f_0') > ... > \exp(f_{n - 1}')$, \eqref{eq:fmfiveexp} yields
      \begin{equation}
        \bigl|\sum\limits_{j = m + 5}^{n - 1} f_j\bigr| \leq \sum\limits_{j = m + 5}^{n - 1} |f_j| < \sum\limits_{j = m + 5}^{n - 1} 2 \cdot 2^{\exp(f_m') - p - 2 - (j - (m + 5))} < \sum\limits_{j = 0}^{\infty} 2^{\exp(f_m') - p - 1 - j} = \epsilon 2^{\exp(f_m')}
        \label{eq:boundfmfivesum}
      \end{equation}

      Using \eqref{eq:sbound} and \eqref{eq:boundfmfivesum},
      \begin{equation}
        \bigl|\sum\limits_{j = m + 5}^{n - 1} f_j\bigr| \leq \sum\limits_{j = m + 5}^{n - 1} |f_j| < \frac{\epsilon}{1 - 4  \sqrt\epsilon}|S_{n - 1}|
        \label{eq:relsboundfmfivesum}
      \end{equation}

      By \eqref{eq:smfiveerror} and \eqref{eq:relsboundfmfivesum}
      \begin{align}
        \left|S_{n-1} - \overline{S_{m+4}}\right|
        & \leq |S_{n-1} - S_{m+4}| + \left|S_{m+4} - \overline{S_{m+4}}\right| \nonumber \\
        & \leq \bigl|\sum_{j=m+5}^{n-1} f_j\bigr| + \frac{6\epsilon}{1-6\sqrt{\epsilon}} |S_{n-1}| \nonumber \\
        & \leq \frac{\epsilon}{1 - 4 \sqrt\epsilon}|S_{n-1}| + \frac{6\epsilon}{1-6\sqrt{\epsilon}} |S_{n-1}| \nonumber \\
        & <  \frac{7\epsilon}{1-6\sqrt{\epsilon}} |S_{n-1}|.
        \label{eq:smfiveerror-1}
      \end{align}

      When combined with \eqref{eq:sbound} this gives
      \begin{align*}
        \left|\overline{S_{m+4}}\right|
        & \geq \left(1-\frac{7 \epsilon}{1-6\sqrt{\epsilon}}\right) |S_{n-1}| \\
        & > \left(1-\frac{7 \epsilon}{1-6\sqrt{\epsilon}}\right) \left(1-4\sqrt{\epsilon}\right) 2^{\exp(f'_m)} \\
        & > \left(1-4\sqrt{\epsilon} - \frac{7 \epsilon \left(1-4\sqrt{\epsilon}\right)}{1-6\sqrt{\epsilon}}\right) 2^{\exp(f'_m)}
      \end{align*}

      which, assuming $\epsilon \ll 1$, can be simplified to
      \begin{equation}
        \left|\overline{S_{m + 4}}\right| > 2^{\exp(f_m') - 1}
        \label{eq:minsmfoursimple}
      \end{equation}

      Using  \eqref{eq:fmfiveexp}, for all $j \geq m + 5$ we have
      \begin{equation}
        |f_j| < 2 \cdot 2^{\exp(f_j')} \leq 2 \cdot 2^{\exp(f_m') - p - 2} = \epsilon \cdot 2^{\exp(f_m') - 1}
        \label{eq:maxfmfive}
      \end{equation}

      And by \eqref{eq:maxfmfive} and \eqref{eq:minsmfoursimple}, all additions
      after $f_{m + 4}$ have no effect (since we are rounding to nearest)
      and we have $\overline{S_{n-1}} = \overline{S_{m+4}}$.
      This, together with \eqref{eq:smfiveerror-1}, implies
      \begin{equation*}
        \left|S_{n-1} - \overline{S_{n-1}}\right| < \frac{7\epsilon}{1-6\sqrt{\epsilon}} |S_{n-1}|
      \end{equation*}
      The proof is complete.
    \end{proof}

    \cite{repsum} discusses the absolute error between the indexed sum and the true sum, but does not
    give a method to compute a floating point approximation of the indexed sum. No error
    bound on the final floating point answer
    was given. Theorem \ref{thm:error} extends the error bound of \cite{repsum} all the way to
    the final return value of the algorithm.

    \begin{thm}
      Consider the $K$-fold indexed sum $Y$ of finite floating point numbers $x_0, \ldots, x_{n - 1}$.
      We denote the true sum $\sum \limits_{j = 0}^{n - 1} x_j$ by $T$, the true
      value of the indexed sum as obtained using \eqref{eq:indexedvalue} by
      $\mathcal{Y}$, and the floating point approximation of $\mathcal{Y}$
      obtained using an appropriate algorithm from Section
      \ref{sec:primitiveops_conv2float} (Algorithm \ref{alg:conv2float} or \ref{alg:conv2floatoverflow}) by $\overline{\mathcal{Y}}$. Assuming the final answer does not overflow,

      \begin{equation}
        \left|T - \overline{\mathcal{Y}}\right| < \left(1 + \frac{7\epsilon}{1 - 6\sqrt\epsilon}\right) \Bigl(n \cdot \max\bigl(2^{W (1 - K)} \max|x_j|, 2^{e_{\min} - 1}\bigr)\Bigr) + \frac{7\epsilon}{1 - 6\sqrt\epsilon} |T|
        \label{eq:error}
      \end{equation}
      \label{thm:error}
    \end{thm}

    \begin{proof}
      The case of all zero input data is trivial, therefore we assume that
      $\max|x_j|$ is nonzero.
      We also assume here no overflow or underflow.
      Let $I$ be the index of $Y$, which is also the index of $\max|x_j|$,
      so that $2^{b_I} > \max|x_j| \geq 2^{a_I}$.
      Therefore for all $i < I$ the slice of any $x_j$ in bin $i$ is $d(x_j, i) = 0$.
      The index of the smallest bin of $Y$ is $I + K - 1$.
      According to Theorem~\ref{thm:dround}, we have
      \begin{align*}
          |x_j - \sum_{i=I}^{I + K -1} d(x_j,i)|
              & = |x_j - \sum_{i=0}^{I + K - 1} d(x_j,i)|
              \leq 2^{a_{I + K -1}} 
              = 2^{a_I - (K-1)W} \\
              & \leq 2^{W(1-K)} \max|x_j|. 
      \end{align*}

      Since the summation in each bin $Y_i$ is exact, we have
      \begin{align}
          |T - \mathcal{Y}| & = |\sum_{j=0}^{n-1} x_j - \sum_{i=I}^{I+K-1} \sum_{j=0}^{n-1} d(x_j, i)|
              = |\sum_{j=0}^{n-1} (x_j - \sum_{i=I}^{I+K-1} d(x_j,i))| \nonumber \\
              & \leq n 2^{W(1-K)} \max|x_j|.
              \label{eq:repboundnaive}
      \end{align}

      However, this bound does not consider underflow. By
      \eqref{eq:droundunderflow}, a small modification yields a bound that
      considers underflow
      \begin{equation}
        \label{eq:repbound}
        |T - \mathcal{Y}| < n \cdot \max\bigl(2^{W  (1 - K)} \max|x_j|, 2^{e_{\min} - 1}\bigr)
      \end{equation}

      By  \eqref{eq:gammadecreases} and \eqref{eq:gammadecreasesfast},
      Theorem \ref{thm:mysortsum} applies to yield
      \begin{equation*}
        \left|\mathcal{Y} - \overline{\mathcal{Y}}\right| < \frac{7\epsilon}{1 - 6\sqrt\epsilon}|\mathcal{Y}|
      \end{equation*}

      By the triangle inequality
      \begin{equation*}
        |\mathcal{Y}| \leq |T| + |T - \mathcal{Y}| < n \cdot \max\bigl(2^{W(1-K)}  \max|x_j|, 2^{e_{\min} - 1}\bigr) + |T|
      \end{equation*}

      The above results can be used to obtain  \eqref{eq:error}, the absolute
      error of the floating point approximation of an indexed sum $|T - \overline{\mathcal{Y}}|$.
      \begin{align}
        \left|T - \overline{\mathcal{Y}}\right| &\leq |T - \mathcal{Y}| + \left|\mathcal{Y} - \overline{\mathcal{Y}}\right| \nonumber \\
        &< n \cdot \max\bigl(2^{W  (1 - K)}  \max|x_j|, 2^{e_{\min} - 1}\bigr) + \frac{7\epsilon}{1 - 6\sqrt\epsilon} |\mathcal{Y}| \nonumber \\
        &< n \cdot \max\bigl(2^{W  (1 - K)}  \max|x_j|, 2^{e_{\min} - 1}\bigr) \nonumber \\
        &+ \frac{7\epsilon}{1 - 6\sqrt\epsilon} \Bigl(n \cdot \max\bigl(2^{W  (1 - K) - 1}  \max|x_j|, 2^{e_{\min} - 1}\bigr) + |T|\Bigr) \nonumber \\
        &< \left(1 + \frac{7\epsilon}{1 - 6\sqrt\epsilon}\right) \Bigl(n \cdot \max\bigl(2^{W (1 - K)} \max|x_j|, 2^{e_{\min} - 1}\bigr)\Bigr) + \frac{7\epsilon}{1 - 6\sqrt\epsilon} |T| \nonumber
      \end{align}
    \end{proof}

    Equation \eqref{eq:error} can be approximated as \eqref{eq:errorapprox}:
    \begin{align}
      \left|T - \overline{\mathcal{Y}}\right| &< \left(1 + \frac{7\epsilon}{1 - 6\sqrt\epsilon}\right) \Bigl(n \cdot \max\bigl(2^{W (1 - K)} \max|x_j|, 2^{e_{\min} - 1}\bigr)\Bigr) + \frac{7\epsilon}{1 - 6\sqrt\epsilon} |T| \nonumber \\
    &\approx n 2^{W  (1 - K)} \max|x_j| + 7  \epsilon |T|
      \label{eq:errorapprox}
    \end{align}

    A perhaps more useful mathematical construction is the error expressed
    relative to the result $\overline{\mathcal{Y}}$, and not the theoretical
    sum $T$. Again by the triangle inequality,
    \begin{equation*}
      |\mathcal{Y}| \leq \left|\overline{\mathcal{Y}}\right| + \left|\mathcal{Y} - \overline{\mathcal{Y}}\right|
    \end{equation*}

    Applying the bound on $|\mathcal{Y} - \overline{\mathcal{Y}}|$ yields
    \begin{equation*}
      |\mathcal{Y}| < \left|\overline{\mathcal{Y}}\right| + \frac{7\epsilon}{1 - 6\sqrt\epsilon}|\mathcal{Y}|
    \end{equation*}

    After simplification,
    \[
      |\mathcal{Y}| < \left(\frac{1}{1 - \frac{7\epsilon}{1 - 6\sqrt\epsilon}}\right)  \left|\overline{\mathcal{Y}}\right| \nonumber 
      = \frac{1 - 6 \sqrt\epsilon}{1 - 6 \sqrt \epsilon - 7\epsilon}  \left|\overline{\mathcal{Y}}\right|.
    \]

    The above results can be used to obtain  \eqref{eq:error2}, the absolute
    error of the floating point approximation of an indexed sum $|T -
    \overline{\mathcal{Y}}|$.
    \begin{align}
      \left|T - \overline{\mathcal{Y}}\right| &\leq |T - \mathcal{Y}| + \left|\mathcal{Y} - \overline{\mathcal{Y}}\right| \nonumber \\
      &< n \cdot \max\bigl(2^{W(1-K)}  \max|x_j|, 2^{e_{\min} - 1}\bigr) + \frac{7\epsilon}{1 - 6\sqrt\epsilon}|\mathcal{Y}| \nonumber \\
      &< n \cdot \max\bigl(2^{W(1-K)}  \max|x_j|, 2^{e_{\min} - 1}\bigr) + \frac{7\epsilon}{1 - 6\sqrt\epsilon}\left(\frac{1 - 6 \sqrt\epsilon}{1 - 6 \sqrt \epsilon - 7\epsilon}\left|\overline{\mathcal{Y}}\right|\right) \nonumber \\
      &= n \cdot \max\bigl(2^{W(1-K)}  \max|x_j|, 2^{e_{\min} - 1}\bigr) + \frac{7\epsilon}{1 - 6 \sqrt \epsilon - 7\epsilon}\left|\overline{\mathcal{Y}}\right|
      \label{eq:error2}
    \end{align}
    \eqref{eq:error2} can be evaluated in ReproBLAS with the \texttt{idxd\_xibound} function in \texttt{idxd.h} (see Section \ref{sec:reproBLAS} for details).

    Equation \eqref{eq:error2} can be approximated as \eqref{eq:error2approx},
    which is nearly equal to bound \eqref{eq:errorapprox}:
    \begin{align}
      |T - \overline{\mathcal{Y}}| &< n \cdot \max\bigl(2^{W(1-K)}  \max|x_j|, 2^{e_{\min} - 1}\bigr) + \frac{7\epsilon}{1 - 6 \sqrt \epsilon - 7\epsilon}  \left|\overline{\mathcal{Y}}\right| \nonumber \\
      &\approx n  2^{W(1 - K)} \max|x_j|+ 7 \epsilon \left|\overline{\mathcal{Y}}\right|
      \label{eq:error2approx}
    \end{align}

    We can compare  \eqref{eq:errorapprox} to the error bound obtained if the
    accumulator fields were summed without extra precision. In this case, only
    the standard summation bound from \cite{higham} would apply and the
    absolute error would be bounded by
    \begin{equation*}
    n \cdot \max\bigl(2^{W(1-K)}  \max|x_j|, 2^{e_{\min} - 1}\bigr) + \left(\frac{(2  K - 1)  \epsilon}{1 - (2  K - 1)  \epsilon}\right)  \sum\limits_0^{2  K - 1}|\gamma_j|
    \end{equation*}
    which is approximately bounded by
    \begin{equation}
    n \cdot \max|x_j| \bigl(2^{W(1-K)} + (2  K - 1)  \epsilon\bigr)
    \label{eq:baderrorapprox}
    \end{equation}

    This is not as tight a bound as \eqref{eq:errorapprox}, and grows linearly
    as the user increases $K$ in an attempt to increase accuracy.
    As depicted in Figure~\ref{fig:conversionmotivation},
    bound \eqref{eq:errorapprox} can be better by over 8 orders of magnitude
    in default configuration for double precision, where $K=3$ and $W = 40$.
