\section{Introduction}
   
  Reproducibility is important for several reasons. Primarily, bitwise identical results are useful and sometimes required for testing and debugging. It is difficult to write tests for a function when the output of the function is not well-defined. Debugging a program becomes very difficult when runs with errors cannot be examined because they cannot be reproduced.

  Reproducibility is also important in normal operations of programs. Simulations looking rare events must be reproducible so that when a rare event is found it may be replayed and studied in more detail. Similarly, forward unstable simulations (such as N-body simulation or climate and weather modeling) must be reproducible as small changes at initial time steps can lead to big changes at later timesteps. Reproduciblity is also required for contractual reasons, when multiple parties must agree on the result of a simulation. Such a situation can arise when multiple runs of a simulation must agree (to ensure that no errors have occured during a run). This situation also arises in the case of multiplayer gaming, when minimal data is sent between computers (to reduce network delays) but all players must agree on the state of the game (so the computation on each players machine must be exact).

  %note: an alternate definition of reproducibility can be used here. We can say that a procedure that computes the result of some function is "reproducible" if the procedure produces equivalent results when the function does.
  At this point is is important to define clearly what we mean by reproduciblity. A computation is \textbf{reproducible} if it achieves bitwise identical results from the same computation when given equivalent inputs. The term ``equivalent inputs'' is intentionally vague, as what constitutes equivalence of inputs depends on the context and computation in question. In this work, we focus on equivalence under the associativity of addition (as this tends to be the main source of non-reproducibility in most computations). 
  %A computation is \textbf{completely reproducibile} if it produces bitwise identical output when given any two inputs equivalent up to permutation with regard to the associativity of addition. A computation is \textbf{block reproducible} if it produces bitwise identical output given any

  Reproducibility is not the same as accuracy, but they are related. Increasing the accuracy of a computation can reduce the probability of nonreproducibility, but it does not guarantee reproducibility, especially for ill-conditioned inputs or when the result is half-way bewteen two floating point numbers in the output precision \cite{taufer}. Ill-conditioned inputs (a tiny sum resulting from extensive cancellation) may be the most common case in many applications. For example, when solving $Ax = b$ using an iterative method, the goal is to get as much cancellation in the residual $r = Ax - b$ as possible.

  With these considerations in mind, reproducible algorithms for summations were presented in \cite{repsum}. The algorithms were shown to enjoy the following properties:
  \begin{enumerate}
    \item They compute a reproducible sum independent of the order of the summands, how they are assigned to processors, or how they are aligned in memory.
    \item They make only basic assumptions about the underlying arithmetic (A subset of IEEE Standard 754-2008).
    \item They scale well as a performance-optimized, non-reproducible implementation, as $n$ (number of summands) and $p$ (number of processors) grow.
    \item The user can choose the desired accuracy of the result. In particular, getting a reproducible result with about the same accurace as the performance optimized algorithm is only be a small constant times slower, but higher accuracy is possible too.
  \end{enumerate}
  The algorithms of \cite{repsum} did not, however, work for all ranges of floating-point input nor did they contain all of the functional components necessary to be deployed in an existing application (they lacked an algorithm to convert from a reproducible type back to floating point format while maintaining accuracy). Our goal is to modify the algorithms in \cite{repsum} so that in addition to the above properties, they enjoy the following properties:
  \begin{enumerate}
    \item The algorithms can be applied in any binary IEEE 754-2008 floating point format, and any valid input in such a format can be summed using the algorithms. The algorithms must be able to sum numbers close to zero, numbers close to overflow, and exceptional values such as \texttt{NaN} and \texttt{Inf}, mimiking the behavior of IEEE floating point wherever possible.
    \item They must be expressed as basic operations that can be applied in several applications. They must be able to be built into a user-friendly, performant library.
  \end{enumerate}

  %TODO insert here a summary of what is to come in each section.
