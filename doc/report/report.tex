\documentclass[12pt]{article}
\usepackage{amsfonts}
\usepackage{listings}
\usepackage{amsmath}
\usepackage{amsthm}
\usepackage{algorithmicx}
\usepackage{algpseudocode}
\usepackage{graphicx}
\usepackage{tikz}
\pagestyle{myheadings}

\author{Peter Ahrens}
\title{ReproBLAS: Much Summation! So Reproducible!}
\providecommand{\ceil}[1]{\left \lceil #1 \right \rceil }
\providecommand{\floor}[1]{\left \lfloor #1 \right \rfloor }
\providecommand{\R}{\ensuremath{\mathbb{R}}}
\providecommand{\F}{\ensuremath{\mathbb{F}}}
\providecommand{\Z}{\ensuremath{\mathbb{Z}}}
\providecommand{\exp}{\ensuremath{\text{exp}}}
\providecommand{\min}{\ensuremath{\text{min}}}
\providecommand{\max}{\ensuremath{\text{max}}}
\providecommand{\ulp}{\ensuremath{\text{ulp}}}
\providecommand{\ufp}{\ensuremath{\text{ufp}}}
\providecommand{\fl}{\ensuremath{\text{fl}}}
\theoremstyle{plain}
\newtheorem{thm}{Theorem}[section]
\newtheorem{lem}[thm]{Lemma}
\newtheorem{alg}{Algorithm}[section]
\newtheorem{eq}{Equation}[section]
\graphicspath{{plots}}
\begin{document}
\noindent
\maketitle
\newpage
\section{Introduction}

\section{Notation and Background}
  Let $\R, \F, $ and $\Z$ denote the sets of real numbers, floating-point numbers, and integers respectively.

  Assume that floating point arithmetic complies with the IEEE 754-2008 standard in "to nearest even" rounding mode and that underflow occurs gradually.

  Let $f = s \times 2^e \times m \in \F$ be a floating-point number represented in IEEE 754 format where $s = \pm 1$ is the \textbf{sign}, $e_{\max} \geq e \geq e_{\min}$ is the \textbf{exponent} ($\exp(f)$ is defined to be $e$), $p$ is the \textbf{precision}, and $m = m_0.m_1m_2...m_{p-1}$ where $m_1, ..., m_{p - 1} \in \{0, 1\}$ is the \textbf{significand} of $f$. $f$ is said to be \textbf{normalized} if $m_0 = 1$ and $e > e_{\min}$, \textbf{unnormalized} if $m_0 = 0$, and \textbf{subnormal} if $m_0 = 0$ and $e = e_{\min}$. $f = 0$ if $m_0 = m_1 = ... = m_{p - 1} = 0$ and $e = e_{\min}$.

  As ReproBLAS is written in C, \verb|float| and \verb|double| refer to the floating point types specified in the 1989 C standard \cite{c89} and we assume that they correspond to the 32-bit and 64-bit types in the IEEE-754-2008 floating point standard \cite{ieee754}. 

  All indicies start at 0 to increase correspondence with the actual ReproBLAS implementation.

\section{The Indexed Type}
  \subsection{Representation}
    The \textbf{indexed type} is used to represent the intermediate result of accumulation using algorithms 6 and 7 specified by Nguyen and Demmel in "Parallel Reproducible Summation" \cite{repsum}.
    An indexed type $Y$ is composed of several accumulators $Y_0, ..., Y_{n - 1} \in \F$, each holding the sum of input in a predetermined exponent range (referred to as a \textbf{bin}). An indexed type with $n$ accumulators is referred to as having \textbf{fold} $n$.

    Let $Y$ be the indexed type corresponding to the reproducibly computed sum of $x_0, ..., x_{n - 1} = s_0 \times 2^{e_0} \times m_0, ..., s_{n - 1} \times 2^{e_{n - 1}} \times m_{n - 1} \in F$.
    If the accumulator $Y_j$ accumulates input in the bin $[a, b), a > b \in \Z$, then $Y_j$ is the sum of all input bits with exponents in $[a, b)$. Specifically, $Y_j = \sum \{s_i2^{e_i}{m_i}_k, i \in \{0, ..., n - 1\}, k \in \Z | e_i - k \in [a, b)\}$.

    For ease of computation, accumulator ranges are not arbitrary. They are broken up into precomputed, fixed width intervals starting at $e_{\max}$. The possible bins $[a_0, b_0), [a_1, b_1), ...$ for an indexed type of \textbf{width} $W$ are $[e_{\max}, e_{\max} - W), [e_{\max} - W, e_{\max} - 2W), ..., [e_{\min} - p - (e_{\max} - e_{\min} + p \mod W) + W, e_{\min} - p - (e_{\max} - e_{\min} + p \mod W))$. It will be explained later why input in the bin $[e_{\min} - p + (e_{\max} - e_{\min} + p \mod W), e_{\min} - p)$ is not accumulated. As discussed in \cite{repsum}, $W < p - 2$. As discussed in section \ref{sec:overflow}, $2W > p$.

    The accumulators in an indexed type correspond to contiguous bins in decreasing order. The \textbf{index} of $Y$ is the smallest $i \in \Z$ such that $2^(a_i + 1) > \max{|x_0|, ..., |x_n|}$ (equivalently, the smallest $i \in \Z$ such that $a_i \geq \max{e_0, ..., e_n}$). If $Y$ has index $i$, then $Y_j, j \in \{0, ..., fold - 1\}$ accumulates numbers in the bin $[a_{i + j}, b_{i + j})$.
  \subsection{Implementation}
    \subsubsection{Data Types}
      As discussed in \cite{repsum}, indexed types are represented using floating point numbers to minimize traffic between floating point and integer arithmetic units. If an indexed type is used to sum \verb|double|, then it is composed entirely of \verb|double| and likewise for \verb|float|. ReproBLAS supports complex types as pairs of real and imaginary components (stored contiguously in memory). If an indexed type is used to sum complex \verb|double| or \verb|float|, then it is composed of pairs of real and imaginary \verb|double| or \verb|float| respectively. The decision to keep the real and imaginary components together (as opposed to keeping separate indexed types for real and imaginary parts of the sum) was motivated by a desire to process accumulators simultaneously with vectorized (SIMD, single instruction multiple data) instructions.
    \subsubsection{Primary and Carry}
      An indexed type's accumulators are each implemented using two underlying floating point fields. The \textbf{primary} field is used during accumulation, while the \textbf{carry} field holds overflow from the primary field. When discussing the primary and carry fields of the accumulator $Y$, we denote the primary fields by $Y_P$ and the carry fields by $Y_C$, so that ${Y_P}_j$ is the primary field of the accumulator $Y_j$. This reflects the fact that the primary fields are stored separately from the carry fields as the primary fields are frequenly accessed sequentially.

      The primary field ${Y_P}_j$ of an indexed type $Y$ of index $i$ and fold $fold$ is stored as an offset from $1.5\times \epsilon^{-1} \times 2^{b_{i + j}}$. This greatly simplifies the process of extracting the components of the input belonging in bin $[a_i, b_i)$.
      As discussed in \cite{repsum}, if we store each primary bin ${Y_P}_j$ as an offset from $1.5 \times \epsilon^{-1} \times 2^{b_{i + j}}$ and keep ${Y_P}_j$ within the range $(2 \times \epsilon^{-1} \times 2^{b_{i + j}}, \epsilon^{-1} \times 2^{b_{i + j}})$, algorithm \ref{alg:deposit} extracts the components of $x$ belonging to bins $[a_i, b_i), ..., [a_{i + fold - 1}, b_{i + fold - 1})$ and adds them to ${Y_P}_0, ..., {Y_P}_{fold - 1}$ without error (and hence reproducibly) for all $x \in \F, |x| < 2 * 2^{a_i}$.
      \begin{alg}
        Extract components of floating point $x$ in bins $[a_i, b_i), ..., [a_{i + fold - 1}, b_{i + fold - 1})$ and add to indexed $Y$. Here, $(r | 1)$ represents the result of setting the last bit of the mantissa ($m_{p - 1}$) of floating-point $r$ to 1.
        \begin{algorithmic}
          \Function{Deposit}{fold, Y, x}
            \State $r \gets x$
            \For{$i = 0 \to (fold - 2)$}
              \State $S \gets {Y_P}_i + (r | 1)$
              \State $q \gets S - {Y_P}_i$
              \State ${Y_P}_i \gets S$
              \State $r \gets r - q$
            \EndFor
            \State ${Y_P}_{fold - 1} \gets {Y_P}_{fold - 1} + (r | 1)$
          \EndFunction
        \end{algorithmic}
        \label{alg:deposit}
      \end{alg}
      The last bit of $r$ is set to break ties when rounding "to nearest even" such that the amount accumulated does not depend on the size of ${Y_P}i$ so far and the same amount is always deposited in all of $Y$'s primary fields regardless of how full they are already. (A dependency on the amount stored in $Y$'s primary fields would imply that the routine is not reproducible. Adding numbers in different orders would accumulate different amounts into the primary fields $Y_P$ when each input is added to them.)

      In order to keep the primary bins in the necessary range during a deposit operation and to keep the representation of $Y$ unique, ${Y_P}_j$ is routinely renormalized to the range $(1.75 \times\epsilon^{-1} \times2^{b_{i + j}}, 1.5 \times\epsilon^{-1} \times2^{b_{i + j}}]$.
      To renormalize, ${Y_P}_j$ is adjusted by $0.25 \times\epsilon^{-1} \times 2^{b_{i + j}}$ when necessary, leaving the carry field ${Y_C}_j$ to record the number of such adjustments. Therefore the value represented by an indexed type $Y$ of index $i$ and fold $fold$ (the sum of the accumulators) can be expressed as in equation \ref{eq:indexedvalue}.
      \begin{eq} The real value $y \in \R$ of an indexed type $Y$

        $y = \sum\limits_{j = 0}^{fold - 1} ({Y_P}_j - 1.5 \times\epsilon^{-1}\times 2^{b_{i + j}}) + {Y_C}_j \times 0.25\times\epsilon^{-1}\times2^{b_{i + j}}$
        \label{eq:indexedvalue}
      \end{eq}
    \subsubsection{Overflow}
      \label{sec:overflow}
      If an indexed type $Y$ has index 0 and the width is $W$, then the value in the primary field ${Y_P}_0$ is stored as an offset from $1.5\times\epsilon^{-1}\times2^{e_{\max} - W} = 1.5 \times 2^{e_{\max} + (p - W)} >= 2 * 2^{e_max}$ since $W < p - 2$ \cite{repsum}, so it is out of the range of the floating-point system and not representable. Before discussing the solution to this overflow problem, take note of theorem \ref{thm:overflow}.
      \begin{thm}
        If $2W > p$, then for any indexed type $Y$ of index $i$ and for any ${Y_P}_j$ such that $i + j \geq 1$, $|{Y_P}_j| < 2^{e_{\max}}$.

        $b_1 = e_{\max} - 2W$ by definition, therefore $b_1 < e_{\max} - p$ and since all quantities are integers, $b_1 \leq e_{\max} - 1 - p$. As $b_0, b_1, ...$ is a positive decreasing sequence, $b_{i + j} \leq e_{\max} - 1 - p$ since $i + j \geq 1$.

        The ${Y_P}_j$ are kept within the range $(2 \times \epsilon^{-1} \times 2^{b_{i + j}}, \epsilon^{-1} \times 2^{b_{i + j}})$, therefore 
        $|{Y_P}_j| < 2 \times \epsilon^{-1} \times 2^{b_{i + j}} \leq 2^{1 + p} \times 2^{e_{\max} - 1 - p} = 2^{e_{\max}}$.

        \label{thm:overflow}
      \end{thm}

    \subsubsection{Underflow}
    \subsubsection{Exceptions}
\section{Basic Operations}
  \subsection{Update}
  \subsection{Deposit}
  \subsection{Renormalize}
  \subsection{Reduce}
  \subsection{Convert}
  \subsection{Error Bound}
  \subsection{Limits}
    \label{sec:limits}
    As absolute value of individual quantities added to ${Y_P}_j$ are not in excess of $2^W b_{i + j}$, $0.25\epsilon^{-1}2^{-W}$ elements may be deposited into ${Y_P}_j$ between renormalizations. As ${Y_C}_j$ must be able to record additions of absolute value 1 without error, ${Y_C}_j$ must stay in the range $(\epsilon^{-1}, -\epsilon^{-1})$. As each renormalization results in addition not in excess absolute value of 2 to ${Y_C}_j$, a maximum of $0.5 * (\epsilon^(-1) - 1)$ renormalizations may be performed, meaning that.
\section{Interface}
  \subsection{Fold}
  \subsection{Complex Types}
  \subsection{Naming Conventions}
  \subsection{Build System}
\section{indexed.h}
  \subsection{Overview}
  \subsection{Types}
  \subsection{Functions}
    \subsubsection{xixadd}
    \subsubsection{xixiadd}
    \subsubsection{xscale}
    \subsubsection{xixiaddsq}
\section{idxBLAS.h}
  \subsection{Overview}
  \subsection{Functions}
    \subsubsection{xixdot}
    \subsubsection{xixnrm2}
    \subsubsection{xixgemv}
    \subsubsection{xixgemm}
  \subsection{Optimization}
\section{repBLAS.h}
\section{idxMPI.h}
  \subsection{Types}
  \subsection{Functions}
    \subsubsection{XIREDUCE}
\section{Applications}
  \subsection{Parallel Reproducible Dot Product}
  \subsection{Parallel Reproducible Vector Norm}
  \subsection{Parallel Reproducible Matrix-Vector Multiply}
  \subsection{Parallel Reproducible Matrix-Matrix Multiply}
\begin{thebibliography}{9}
  \bibitem{sortingsummation}
    Demmel, James W., and Yozo Hida. Accurate floating point summation. Computer Science Division, University of California, 2002.
  \bibitem{repsum}
    Demmel, James, and Hong Diep Nguyen. Parallel Reproducible Summation. IEEE Transactions on Computers, 2014.
  \bibitem{ieee754}
    IEEE Standard for Floating-Point Arithmetic," IEEE Std 754-2008 , vol., no., pp.1,70, Aug. 29 2008
  \bibitem{c89}
    ANSI/ISO 9899-1990 American National Standard for Programming Languages - C, section 6.1.2.5
\end{thebibliography}
\end{document}
