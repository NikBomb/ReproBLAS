\documentclass[12pt]{article}
\usepackage{amsfonts}
\usepackage{listings}
\usepackage{amsmath}
\usepackage{amsthm}
\usepackage{algorithmicx}
\usepackage{algpseudocode}
\usepackage{graphicx}
\usepackage{tikz}
\pagestyle{myheadings}

\author{Peter Ahrens}
\title{ReproBLAS: Much Summation! So Reproducible!}
\providecommand{\ceil}[1]{\left \lceil #1 \right \rceil }
\providecommand{\floor}[1]{\left \lfloor #1 \right \rfloor }
\providecommand{\R}{\ensuremath{\mathbb{R}}}
\providecommand{\F}{\ensuremath{\mathbb{F}}}
\providecommand{\Z}{\ensuremath{\mathbb{Z}}}
\providecommand{\exp}{\ensuremath{\text{exp}}}
\providecommand{\min}{\ensuremath{\text{min}}}
\providecommand{\max}{\ensuremath{\text{max}}}
\providecommand{\ulp}{\ensuremath{\text{ulp}}}
\providecommand{\ufp}{\ensuremath{\text{ufp}}}
\providecommand{\fl}{\ensuremath{\text{fl}}}
\theoremstyle{plain}
\newtheorem{thm}{Theorem}[section]
\newtheorem{lem}[thm]{Lemma}
\newtheorem{alg}{Algorithm}[section]
\newtheorem{eq}{Equation}[section]
\graphicspath{{plots}}
\begin{document}
\noindent
\maketitle
\newpage
\section{Introduction}

\section{Notation and Background}
  Let $\R, \F, $ and $\Z$ denote the sets of real numbers, floating-point numbers, and integers respectively.

  Assume that floating point arithmetic complies with the IEEE 754-2008 standard in "to nearest even" rounding mode and that underflow occurs gradually.

  Let $f = s \times 2^e \times m \in \F$ be a floating-point number represented in IEEE 754 format where $s = \pm 1$ is the \textbf{sign}, $e_{\max} \geq e \geq e_{\min}$ is the \textbf{exponent} ($\exp(f)$ is defined to be $e$), $p$ is the \textbf{precision}, and $m = m_0.m_1m_2...m_{p-1}$ where $m_1, ..., m_{p - 1} \in \{0, 1\}$ is the \textbf{significand} of $f$. $f$ is said to be \textbf{normalized} if $m_0 = 1$ and $e > e_{\min}$, \textbf{unnormalized} if $m_0 = 0$, and \textbf{subnormal} if $m_0 = 0$ and $e = e_{\min}$. $f = 0$ if $m_0 = m_1 = ... = m_{p - 1} = 0$ and $e = e_{\min}$.

  As ReproBLAS is written in C, \verb|float| and \verb|double| refer to the floating point types specified in the 1989 C standard \cite{c89} and we assume that they correspond to the 32-bit and 64-bit types in the IEEE-754-2008 floating point standard \cite{ieee754}. 

  All indicies start at 0 to increase correspondence with the actual ReproBLAS implementation.

\section{The Indexed Type}
  \subsection{Representation}
    The \textbf{indexed type} is used to represent the intermediate result of accumulation using algorithms 6 and 7 specified by Nguyen and Demmel in "Parallel Reproducible Summation" \cite{repsum}.
    An indexed type $Y$ is composed of several accumulators $Y_0, ..., Y_{n - 1} \in \F$, each holding the sum of input in a predetermined exponent range (referred to as a \textbf{bin}). An indexed type with $n$ accumulators is referred to as having \textbf{fold} $n$. Due to their wild innacuraccy, indexed types of fold 1 are not considered. It is assumed that indexed types have fold greater than 1.

    Let $Y$ be the indexed type corresponding to the reproducibly computed sum of $x_0, ..., x_{n - 1} = s_0 \times 2^{e_0} \times m_0, ..., s_{n - 1} \times 2^{e_{n - 1}} \times m_{n - 1} \in F$.
    If the accumulator $Y_j$ accumulates input in the bin $(a, b], a < b \in \Z$, then $Y_j$ is the sum of all input bits with exponents in $(a, b]$. Specifically, $Y_j = \sum \{s_i2^{e_i}{m_i}_k, i \in \{0, ..., n - 1\}, k \in \Z | e_i - k \in (a, b]\}$.

    For ease of computation, accumulator ranges are not arbitrary. They are broken up into precomputed, fixed width intervals starting at $e_{\max}$. The possible bins

    $(a_0, b_0],$

    $(a_1, b_1],$

    $\vdots$

    for an indexed type of \textbf{width} $W$ are

    $(e_{\max} - W, e_{\max}],$

    $(e_{\max} - 2\times W, e_{\max} - W],$

    $\vdots$

    $(e_{\min} - p + 2 + W - (e_{\max} - e_{\min} + p - 2 \mod W),$

    \indent \indent $e_{\min} - p + 2 + 2 \times W - (e_{\max} - e_{\min} + p - 2 \mod W)]$.

    Section \ref{sec:underflow} explains why input in the bin

    $(e_{\min} - p, e_{\min} - p + 2 + W + (e_{\max} - e_{\min} + p - 2 \mod W)]$

    is not accumulated.

    As discussed in \cite{repsum}, $W < p - 2$. As discussed in section \ref{sec:overflow}, $2\times W > p$.

    The accumulators in an indexed type correspond to contiguous bins in decreasing order. The \textbf{index} of $Y$ is the smallest $i \in \Z$ such that $2^{b_i + 1} > \max(|x_0|, ..., |x_n|)$ (equivalently, the smallest $i \in \Z$ such that $b_i \geq \max{e_0, ..., e_n}$). If $Y$ has index $i$, then $Y_j, j \in \{0, ..., fold - 1\}$ accumulates numbers in the bin $(a_{i + j}, b_{i + j}]$.

  \subsection{Implementation}
    \subsubsection{Data Types}
      As discussed in \cite{repsum}, indexed types are represented using floating point numbers to minimize traffic between floating point and integer arithmetic units. If an indexed type is used to sum \verb|double|, then it is composed entirely of \verb|double| and likewise for \verb|float|. ReproBLAS supports complex types as pairs of real and imaginary components (stored contiguously in memory). If an indexed type is used to sum complex \verb|double| or \verb|float|, then it is composed of pairs of real and imaginary \verb|double| or \verb|float| respectively. The decision to keep the real and imaginary components together (as opposed to keeping separate indexed types for real and imaginary parts of the sum) was motivated by a desire to process accumulators simultaneously with vectorized (SIMD, single instruction multiple data) instructions.

    \subsubsection{Primary and Carry}
      An indexed type's accumulators are each implemented using two underlying floating point fields. The \textbf{primary} field is used during accumulation, while the \textbf{carry} field holds overflow from the primary field. When discussing the primary and carry fields of the accumulator $Y$, we denote the primary fields by $Y_P$ and the carry fields by $Y_C$, so that ${Y_P}_j$ is the primary field of the accumulator $Y_j$. This reflects the fact that the primary fields are stored separately from the carry fields as the primary fields are frequenly accessed sequentially.

      The primary field ${Y_P}_j$ of an indexed type $Y$ of index $i$ and fold $fold$ is stored as an offset from $1.5\times \epsilon^{-1} \times 2^{a_{i + j}}$. This greatly simplifies the process of extracting the components of the input belonging in bin $(a_i, b_i]$.
      The value ${y_P}_j$ stored in the primary field ${Y_P}_j$ of an indexed type $Y$ of index $i$ expressed in equation \ref{eq:pri}
      \begin{eq} The value ${y_P}_j$ stored in the primary field ${Y_P}_j$ of an indexed type $Y$ of index $i$

        ${y_P}_j = {Y_P}_j - 1.5\times\epsilon^{-1}\times2^{a_{i + j}}$
        \label{eq:pri}
      \end{eq}
      As discussed in \cite{repsum}, if we store each primary bin ${Y_P}_j$ as an offset from $1.5 \times \epsilon^{-1} \times 2^{a_{i + j}}$ and keep ${Y_P}_j$ within the range $(\epsilon^{-1} \times 2^{a_{i + j}}, 2 \times \epsilon^{-1} \times 2^{a_{i + j}})$, algorithm \ref{alg:deposit} extracts the components of $x$ belonging to bins $(a_i, b_i], ..., (a_{i + fold - 1}, b_{i + fold - 1}]$ and adds them to ${Y_P}_0, ..., {Y_P}_{fold - 1}$ without error (and hence reproducibly) for all $x \in \F, |x| < 2 * 2^{b_i}$.
      \begin{alg}
        Extract components of floating point $x$ in bins $(a_i, b_i], ..., (a_{i + fold - 1}, b_{i + fold - 1}]$ and add to indexed $Y$. Here, $(r | 1)$ represents the result of setting the last bit of the mantissa ($m_{p - 1}$) of floating-point $r$ to 1.
        \begin{algorithmic}
          \Function{Deposit}{fold, x, Y}
            \State $r \gets x$
            \For{$i = 0 \to (fold - 2)$}
              \State $S \gets {Y_P}_i + (r | 1)$
              \State $q \gets S - {Y_P}_i$
              \State ${Y_P}_i \gets S$
              \State $r \gets r - q$
            \EndFor
            \State ${Y_P}_{fold - 1} \gets {Y_P}_{fold - 1} + (r | 1)$
          \EndFunction
        \end{algorithmic}
        \label{alg:deposit}
      \end{alg}
      The last bit of $r$ is set to break ties when rounding "to nearest even" such that the amount accumulated does not depend on the size of ${Y_P}_i$ so far and the same amount is always deposited in all of $Y$'s primary fields regardless of how full they are already. (A dependency on the amount stored in $Y$'s primary fields would imply that the routine is not reproducible. Adding numbers in different orders would accumulate different amounts into the primary fields $Y_P$ when each input is added to them.)

      In order to keep the primary bins in the necessary range during a deposit operation and to keep the representation of $Y$ unique, ${Y_P}_j$ is routinely renormalized to the range $[1.5 \times\epsilon^{-1} \times 2^{a_{i + j}}, 1.75 \times\epsilon^{-1} \times 2^{a_{i + j}})$.
      To renormalize, ${Y_P}_j$ is adjusted by $0.25 \times\epsilon^{-1} \times 2^{a_{i + j}}$ when necessary, leaving the carry field ${Y_C}_j$ to record the number of such adjustments.
      The value ${y_C}_j$ stored in the carry field ${Y_C}_j$ of an indexed type $Y$ of index $i$ expressed in equation \ref{eq:car}
      \begin{eq} The value ${y_C}_j$ stored in the carry field ${Y_C}_j$ of an indexed type $Y$ of index $i$

        ${y_C}_j = {Y_C}_j \times 0.25\times\epsilon^{-1}\times2^{a_{i + j}}$
        \label{eq:car}
      \end{eq}
 Therefore the value represented by an indexed type $Y$ of index $i$ and fold $fold$ (the sum of the accumulators) can be expressed as in equation \ref{eq:indexedvalue}.
      \begin{eq} The real value $y \in \R$ of an indexed type $Y$ of index $i$

        $y = \sum\limits_{j = 0}^{fold - 1} {y_P}_j + {y_C}_j = \sum\limits_{j = 0}^{fold - 1} ({Y_P}_j - 1.5 \times\epsilon^{-1}\times 2^{a_{i + j}}) + {Y_C}_j \times 0.25\times\epsilon^{-1}\times2^{a_{i + j}}$
        \label{eq:indexedvalue}
      \end{eq}
      It is worth noting here that because the primary field ${Y_P}_0$ is stored with an exponent of $a_i + p$, it is unnecessary to store the index of an indexed type explicitly. The index can be determined by simply examining the exponent of ${Y_P}_0$, as the mapping between the exponent of ${Y_P}_0$ and the index of $Y$ is bijective.

    \subsubsection{Overflow}
      \label{sec:overflow}
      If an indexed type $Y$ has index 0 and the width is $W$, then the value in the primary field ${Y_P}_0$ is stored as an offset from $1.5\times\epsilon^{-1}\times2^{e_{\max} - W}$. However, $1.5\times\epsilon^{-1}\times2^{e_{\max} - W} = 1.5 \times 2^{e_{\max} + (p - W)} > 2 * 2^{e_{\max}}$ since $W < p - 2$ \cite{repsum}, so it is out of the range of the floating-point system and not representable. Before discussing the solution to this overflow problem, take note of theorem \ref{thm:overflow}.
      \begin{thm}
        If $2\times W > p$, then for any indexed type $Y$ of index $i$ and any ${Y_P}_j$ such that $i + j \geq 1$, $|{Y_P}_j| < 2^{e_{\max}}$.

        $a_1 = e_{\max} - 2\times W$ by definition, therefore $a_1 < e_{\max} - p$ and since all quantities are integers, $a_1 \leq e_{\max} - 1 - p$. As $a_0, a_1, ...$ is a positive decreasing sequence, $a_{i + j} \leq e_{\max} - 1 - p$ since $i + j \geq 1$.

        ${Y_P}_j$ is kept within the range $(\epsilon^{-1} \times 2^{a_{i + j}}, 2 \times \epsilon^{-1} \times 2^{a_{i + j}})$, therefore
        $|{Y_P}_j| < 2 \times \epsilon^{-1} \times 2^{a_{i + j}} \leq 2^{1 + p} \times 2^{e_{\max} - 1 - p} = 2^{e_{\max}}$.

        \label{thm:overflow}
      \end{thm}
      By theorem \ref{thm:overflow}, if $2\times W > p$ then the only primary field that could possibly be in overflow is a primary field corresponding to bin 0, and all other primary fields have exponent less than $e_{\max}$. Therefore, we impose $2\times W > p$ and express the value of the primary field corresponding to bin 0 as a scaled offset from $1.5\times2^{e_{\max}}$. Note that this still distinguishes the primary field corresponding to bin 0 because no other primary field has an exponent of $e_{\max}$. The value ${y_P}_0$ of the primary field ${Y_P}_0$ of an indexed type $Y$ of index 0 is expressed in equation \ref{eq:pri0}.
      \begin{eq} The value ${y_P}_0$ of the primary field ${Y_P}_0$ of an indexed type $Y$ of index 0.

        ${y_P}_0 = 2^{p - W}\times({Y_P}_0 - 1.5\times2^{e_{\max}})$
        \label{eq:pri0}
      \end{eq}
      Of course, what remains to be seen is how we can extract and add (as in algorithm \ref{alg:deposit}) the components of a floating point number to an indexed type $Y$ of index 0. Algorithm \ref{alg:deposit0} shows the adaptation of algorithm $\ref{alg:deposit}$ for types of index 0. Note that all of the same assumptions must hold from \ref{alg:deposit} except that ${Y_P}_0$ must now be kept within the range $(2^{e_{\max}}, 2 \times 2^{e_{\max}})$.
      \begin{alg}
        Extract components of floating point $x$ in bins $(a_0, b_0], ..., (a_{fold - 1}, b_{fold - 1}]$ and add to indexed $Y$ of index 0. Here, $(r | 1)$ represents the result of setting the last bit of the mantissa ($m_{p - 1}$) of floating-point $r$ to 1.
        \begin{algorithmic}
          \Function{Deposit0}{fold, x, Y}
            \State $r \gets x / 2^{p - w}$
            \State $S \gets {Y_P}_0 + (r | 1)$
            \State $q \gets S - {Y_P}_0$
            \State ${Y_P}_0 \gets S$
            \State $q \gets q \times 2^{p - w}$
            \State $r \gets x - q$
            \For{$i = 1 \to (fold - 2)$}
              \State $S \gets {Y_P}_i + (r | 1)$
              \State $q \gets S - {Y_P}_i$
              \State ${Y_P}_i \gets S$
              \State $r \gets r - q$
            \EndFor
            \State ${Y_P}_{fold - 1} \gets {Y_P}_{fold - 1} + (r | 1)$
          \EndFunction
        \end{algorithmic}
        \label{alg:deposit0}
      \end{alg}

      Algorithm \ref{alg:deposit0} has a few key features worth pointing out. First, since $x < 2 \times 2^{e_{\max}}$, then the value of $r$ after it's first assignment satisfies $r < 2 \times 2^{e_{\max} - p + W}$.
      Furthermore, since $x$ is scaled by a factor of two, the significand of $r$ is equal to that of $x$.
      Therefore, the component of $r$ belonging to the bin $(e_{\max} - p, e_{\max} - p + W]$ is equivalent to the component of $x$ belonging to the bin $(e_{\max} - W, e_{\max}]$ scaled by $2^{p - W}$. This component of $r$ is extracted, stored in q, and added to ${Y_P}_0$. We can therefore scale this component by $2^{p - w}$ to obtain the component of $x$ belonging to the bin $(e_{\max} - W, e_{\max}]$. As $x$ is representable, so is this component. We can then subtract this component from $x$ and continue as in algorithm \ref{alg:deposit}.

      Notice that after the first extraction, we cannot simply subtract $q$ from $r$ and then scale $r$ up and continue, as there may have been underflow when scaling $x$ down. We must scale $q$ up and subtract it from the original $x$.

    \subsubsection{Underflow}
      \label{sec:underflow}
      Here we consider the effects of underflow on algorithms described in \cite{repsum} and how they can be avoided. In algorithm \ref{alg:deposit}, when the last bit of $r$ is set to 1, it is assumed that this last bit will only serve to fix the direction of the rounding mode. However, if $r$ is a denormalized number, that last bit may very well be accumulated!

      For algorithm \ref{alg:deposit} to work correctly when adding $r | 1$ to ${Y_P}_j$, $|r - (r | 1)|$ must be less than half of the rounding error when adding a number to ${Y_P}_j$. Mathematically, $\ulp(r) < \ulp({Y_P}_j)/2$.
      If $r$ is a normalized number, then because it is assumed $\ufp({Y_P}_j) \geq 2^{p - W}\ufp(r)$ and $W < p - 2$, we have $\ulp(r) < \ulp({Y_P}_j)/2$. Therefore, it is only possible for $\ulp(r) \geq \ulp({Y_P}_j)/2$ if $r$ is denormalized.

      The value of the last bit of a denormalized number is a constant equal to $2 \times 2^{e_{\min} - p}$. Therefore, the smallest bin $[a, b)$ for which algorithm \ref{alg:deposit} works correctly has $b > e_{\min} - p + 2$. The smallest bin is

      $(e_{\min} - p + 2 + W - (e_{\max} - e_{\min} + p - 2 \mod W),$

      \indent \indent $e_{\min} - p + 2 + 2 \times W - (e_{\max} - e_{\min} + p - 2 \mod W)]$.

      and since

      $e_{\min} - p + 2 + W - (e_{\max} - e_{\min} + p - 2 \mod W) > e_{\min} - p + 2$,

      deposits on this bin do not result in errors.

      Since $W < p - 2$,

      $e_{\min} - p + 2 + W - (e_{\max} - e_{\min} + p - 2 \mod W) < e_{\min}$.

      Therefore, at least all of the input in the bin $[e_{\min}, e_{\max}]$ is accumulated.

      There are several ways to accumulate the input in the denormal range. One simple way the algorithm could be extended to denormal inputs would be to scale the bottom bins up, as was done to handle overflow. Another way to accumulate this input would be to simply add it to a final primary field without setting the last bit. Because the denormal number would fit entirely in the final bin, this operation would be performed without error. This would come at the cost of additional branching. Because of the relatively low priority for accumulating denormal values, these methods were not implemented.

    \subsubsection{Exceptions}
      Indexed types are capable of representing exceptional cases such as \verb|NaN| (Not a Number) and \verb|Inf| (Infinity). An indexed type $Y$ stores it's exception status in it's first primary field ${Y_P}_0$.

      A value of $0$ in ${Y_P}_0$ indicates that nothing has been added to ${Y_P}_0$ yet. The smallest bin is

      $(e_{\min} - p + 2 + W - (e_{\max} - e_{\min} + p - 2 \mod W),$

      \indent \indent $e_{\min} - p + 2 + 2 \times W - (e_{\max} - e_{\min} + p - 2 \mod W)]$

      and since the ${Y_P}_j$ are kept within the range $(\epsilon^{-1} \times 2^{a_{i + j}}, 2 \times \epsilon^{-1} \times 2^{a_{i + j}})$ (where $i$ is the index), the smallest possible ${Y_P}_j$ is

      $\epsilon^{-1} \times 2^{e_{\min} - p + 2 + W - (e_{\max} - e_{\min} + p - 2 \mod W)} > 2^{e_{\min}} > 0$.

      Therefore $0$ is previously unused in other contexts and is a valid sentinel value. (As the exponent of $0$ is distinct from the exponent of normalized values, the bijection between the index of an indexed type $Y$ and the exponent of ${Y_P}_0$ is preserved)

      A value of \verb|Inf| or \verb|-Inf| in ${Y_P}_0$ indicates that one or more \verb|Inf| or \verb|-Inf| (and no other exeptional values) have been added to $Y$ respectively.

      A value of \verb|NaN| in ${Y_P}_0$ indicates that one or more \verb|NaN| have been added to $Y$ or one or more of both \verb|Inf| and \verb|-Inf| have been added to $Y$.

      As the ${Y_P}_j$ are kept finite to store finite values, \verb|Inf|, \verb|-Inf|, and \verb|NaN| are previously unused in other contexts and are valid sentinel values. (As the exponent of \verb|Inf|, \verb|-Inf|, and \verb|NaN| is distinct from the exponent of finite values, the bijection between the index of an indexed type $Y$ and the exponent of ${Y_P}_0$ is preserved)

      This behavior follows the behaviour for exceptional values in IEEE 754-2008 floating point arithmetic and is therefore easy to implement. At the beginning of \ref{alg:deposit}, we may simply check $x$ and ${Y_P}_0$ for the exceptional values \verb|Inf|, \verb|-Inf|, and \verb|NaN|. If any one of $x$ or ${Y_P}_0$ is indeed exceptional, we add $x$ to the (possibly finite) ${Y_P}_0$. Otherwise, we can continue adding finite values as we have always done.

      As \verb|Inf|, \verb|-Inf|, and \verb|NaN| add associatively, this behaviour is reproducible.

      It should be noted here that it is very much possible to achieve a final result of \verb|Inf| or \verb|-Inf| when ${Y_P}_0$ is finite. This is due to the fact that the indexed representation can express values outside of the range of the floating point numbers that it is composed with. More specifically, it is very much possible for the value $y$ represented by the indexed type $Y$ to satisfy $|y| \geq 2 * 2^{e_{\max}}$.

\section{Basic Operations}
  \subsection{Update}
    As noted in algorithm \ref{alg:deposit}, when adding $x \in \F$ to an indexed type $Y$ of index $i$ and fold $fold$, we make the assumption that $|x| < 2 * 2^{b_i}$. However, this is not always the case and sometimes it is necessary to adjust the index of $Y$. This adjustment is called an \textbf{update}. The process of updating $Y$ to have index $i' > i$ is summarized succinctly in algorithm \ref{alg:update}.
    \begin{alg}
      Update indexed $Y$ of index $i$ and fold $fold$ to have an index suitable to deposit $x$ where $x$ lies in the bin $(a_{i'}, b_{i'}]$.
      \begin{algorithmic}
        \Function{Update}{fold, x, Y}
          \If{$i' < i$}
            \State $[{Y_P}_{\min(i - i', fold)}, ..., {Y_P}_{fold - 1}] \gets [{Y_P}_0, ..., {Y_P}_{fold - 1 - \min(i - i', fold)}]$
            \State $[{Y_P}_0, ..., {Y_P}_{\min(i - i', fold) - 1}] \gets [1.5 \times \epsilon^{-1} \times a_{i'}, ..., 1.5 \times \epsilon^{-1} \times a_{\min(i, fold + i') - 1}]$
            \State $[{Y_C}_{\min(i - i', fold)}, ..., {Y_C}_{fold - 1}] \gets [{Y_C}_0, ..., {Y_C}_{fold - 1 - \min(i - i', fold)}]$
            \State $[{Y_C}_0, ..., {Y_C}_{\min(i - i', fold) - 1}] \gets [0, ..., 0]$
          \EndIf
        \EndFunction
      \end{algorithmic}
      \label{alg:update}
    \end{alg}
    The update operation is described in the update section of algorithm $6$ in \cite{repsum}.

    It should be noted that if ${Y_P}_0$ is 0, then the update is performed as if $i + fold < i'$. If ${Y_P}_0$ is \verb|Inf|, \verb|-Inf|, or \verb|NaN|, then $Y$ is not modified by an update.

    To speed up this operation, the factors $1.5 \times \epsilon^{-1} \times a_k$ for all valid $k \in Z$ are stored in a precomputed array.

  \subsection{Deposit}
    The deposit operation (here referred to as algorithm \ref{alg:deposit}, a special case where $fold = 3$ is described as \verb|ExtractVector3| in \cite{repsum}) lies firmly at the heart of ReproBLAS.

    The indexed type $Y$ that results from depositing the floating point values $x_0, ..., x_{n - 1}$ into an empty indexed type is referred to as the \textbf{indexed sum} of $x_0, ..., x_{n - 1}$

  \subsection{Renormalize}
    When depositing values into an indexed type $Y$ of index $i$, the assumption is made that ${Y_P}_j \in (\epsilon^{-1}\times 2^{a_{i + j}}, 2 \times \epsilon^{-1}\times 2^{a_{i + j}})$ throughout the routine. To enforce this condition, the indexed type must be renormalized every $2^{p - w - 2}$ deposit operations. The renormalization procedure is shown in algorithm \ref{alg:renorm}.
    \begin{alg}
      For an indexed $Y$ with index $i$, assuming ${Y_P}_j \in [1.25 \times \epsilon^{-1}\times 2^{a_{i + j}}, 2 \times \epsilon^{-1}\times 2^{a_{i + j}})$, renormalize $Y$ such that ${Y_P}_j \in [1.5 \times \epsilon^{-1}\times 2^{a_{i + j}}, 1.75 \times \epsilon^{-1}\times 2^{a_{i + j}})$.
      \begin{algorithmic}
        \Function{Renorm}{fold, Y}
          \For{$j \in 0 \to fold - 1$}
            \If{${Y_P}_j < 1.5 \times \ufp({Y_P}_j)$}
              \State ${Y_P}_j \gets {Y_P}_j + 0.25 \times \ufp({Y_P}_j)$
              \State ${Y_C}_j \gets {Y_C}_j - 1$
            \EndIf
            \If{${Y_P}_j \geq 1.75 \times \ufp({Y_P}_j)$}
              \State ${Y_P}_j \gets {Y_P}_j - 0.25 \times \ufp({Y_P}_j)$
              \State ${Y_C}_j \gets {Y_C}_j + 1$
            \EndIf
          \EndFor
        \EndFunction
      \end{algorithmic}
      \label{alg:renorm}
    \end{alg}
    The renormalization operation is described in the carry-bit propagation section of algorithm $6$ in \cite{repsum}, although it has been slightly modified so as not to include an extraneous case.

    To show the reasoning behind the assumptions in algorithm \ref{alg:renorm}, consider the case in which floating point $x_0, x_1, ...$ are succesively deposited in an indexed type $Y$ which satisfies ${Y_P}_j \in [1.5 \times \epsilon^{-1}\times 2^{a_{i + j}}, 1.75 \times \epsilon^{-1}\times 2^{a_{i + j}})$. (Such a condition is satisfied upon initialization of an empty $Y$)

    Let $d_{j, k}$ denote the amount added to ${Y_P}_j$ with the $k^th$ deposit (without renormalization).

    As the deposit operation extracts and adds the appropriate component of $x_k$ exactly (assuming ${Y_P}_j$ lies within the appropriate boundaries at each step, which will be shown), $d_{j, k}$ is completely contained in the bin $[a_{i + j}, b_{i + j})$, meaning $|d_{j, k}| < 2^{b_{i + j}} = 2^{a_{i + j} + W}$.

    Therefore,

    $|\sum \limits_{n = 0}^{k - 1} d_{j, k}| < \sum \limits_{k = 0}^{n - 1} |d_k| < n \times 2^{a_{i + j} + W}$

    and if $n <= 2^{p - W - 2}$, then after the $n^{th}$ deposit

    ${Y_P}_j \in [(1.5 \times \epsilon^{-1} - n \times 2^W)\times 2^{a_{i + j}}, (1.75 \times \epsilon^{-1} + n \times 2^W)\times 2^{a_{i + j}})$

    \indent\indent$\in [1.25 \times \epsilon^{-1}\times 2^{a_{i + j}}, 2 \times \epsilon^{-1}\times 2^{a_{i + j}})$

  \subsection{Reduce}
    An operation to produce the sum of two indexed types is necessary to perform a reduction. For completeness we include the algorithm here, although apart from the renormalization step, it is equivalent to algorithm $7$ in \cite{repsum}.
    \begin{alg}
      For an indexed type $Y$ (the indexed sum of some $x_0, ..., x_{n - 1} \in \F$) with index $i$ and indexed type $Y'$ with index $i'$ (the indexed sum of some $x_n, ..., x_{n + m - 1} \in \F$), add $Y'$ to $Y$. That is, set $Y$ to the indexed sum of $x_0, ..., x_{n + m - 1}$. At the start and end of the algorithm, we may assume ${Y_P}_j, {Y'_P}_j \in [1.5 \times \epsilon^{-1}\times 2^{a_{i + j}}, 1.75 \times \epsilon^{-1}\times 2^{a_{i + j}})$
      \begin{algorithmic}
        \Function{Reduce}{fold, Y, Y'}
          \If{$i' > i$}
            \State $Y^{\prime\prime} \gets Y'$
            \State \Call{Reduce}{fold, $Y^{\prime\prime}$, $Y$}
            \State $Y \gets Y^{\prime\prime}$
          \EndIf
          \For{$j \in 0 \to i'$}
            \State ${Y_P}_{j + i - i'} \gets {Y_P}_{j + i - i'} + {Y'_P}_j - 1.5 \times \ufp({Y_P}_j)$
            \State ${Y_C}_{j + i - i'} \gets {Y_C}_{j + i - i'} + {Y'_C}_j$
          \EndFor
          \State \Call{Renorm}{fold, $Y$}
        \EndFunction
      \end{algorithmic}
      \label{alg:reduce}
    \end{alg}

  \subsection{Convert}
    At certain points in using the alternate number system provided by ReproBLAS, the user may wish to convert between indexed and floating point representations of a number.

    Converting a floating point number to an indexed type should produce, for transparency and reproducibility, an indexed type equivalent to the indexed sum of the floating point number.
    The procedure is very simply summarized by algorithm \ref{alg:conv2indexed}
    \begin{alg}
      Convert floating point $x$ to indeded $Y$ with fold $fold$.
      \begin{algorithmic}
        \Function{ConvertFloatToIndexed}{fold, x, Y}
          \State $Y \gets 0$
          \State \Call{Deposit}{fold, x, Y}
        \EndFunction
      \end{algorithmic}
      \label{alg:conv2indexed}
    \end{alg}

    Converting an indexed type to a floating point number is more difficult. However, because \cite{repsum} guarantees that all fields in the indexed type are reproducible, the problem comes down to evaluating equation \ref{eq:indexedvalue} accurately and without unnecessary overflow. Following the methods in \cite{sortsum}, we add the floating point numbers in order of decreasing unnormalized exponent using a higher intermediate precision. As discussed in $\cite{sortsum}$, the intermediate precision must be greater than $p + \ceil{\log_2(2 \times fold)}$ as there are $2 \times fold$ elements to be summed.

    When adding the fields it is not necessary to examine the values in the fields or sort them explicitly. Their unnormalized exponent does not depend on their values, and their unnormalized exponents have a predetermined order.

    Consider indexed $Y$ of index $i$ and fold $fold$.
    As each primary field can be expressed as an offset from $1.5 \times \epsilon^{-1} \times 2^{a_{i + j}}$, ${y_P}_j$ can be expressed exactly using an unnormalized floating point type ${y'_P}_j$ with an exponent of $a_{i + j} + p$.
    As each carry field ${Y_C}_j$ is a count of renormalization adjustments later scaled by $0.25 \times \epsilon^{-1} \times 2^{a_{i + j}}$, ${y_C}_j$ can be expressed exactly using an unnormalized floating point number ${y'_C}_j$ with an exponent of $a_{i + j} + p + p - 2$.

    For all $k < j$, $\exp({y'_P}_k) > \exp({y'_P}_j)$ and $\exp({y'_C}_k) > \exp({y'_C}_j)$ because $a_{i + k} > a_{i + j}$.

    $\exp({y'_C}_j) = a_{i + j} + p + p - 2$ and $\exp({y'_P}_{j - 1}) = a_{i + j - 1} + p = a_{i + j} + p + W$. Therefore $\exp({y'_C}_j) > \exp({y'_P}_{j - 1})$ using $W < p - 2$.

    $\exp({y'_C}_j) = a_{i + j} + p + p - 2$ and $\exp({y'_P}_{j - 2}) = a_{i + j - 1} + p = a_{i + j} + p + 2\times W$. Therefore $\exp({y'_C}_j) < \exp({y'_P}_{j - 2})$ using $2W > p$.

    Therefore, $\exp({y'_C}_0) > \exp({y'_C}_1) > \exp({y'_P}_0) > \exp({y'_C}_2) > \exp({y'_P}_1) > ... > \exp({y'_C}_{fold - 2}) > \exp({y'_P}_{fold - 3}) > \exp({y'_C}_{fold - 1}) > \exp({y'_P}_{fold - 2}) > \exp({y'_P}_{fold - 1})$. These numbers may, for convenience of notation, be referred to in order as $y'_0, ..., y'_{2 \times fold - 1}$

    It should be noted that the ${y'_P}_j$ and the ${y'_C}_j$ can be expressed exactly using the same floating point types as ${Y_P}_j$ and ${Y_C}_j$ (except in the case of overflow, in which a scaled version may be obtained), and such exact floating point representations can be obtained using equations \ref{eq:pri} and \ref{eq:car}.

    As $|{y'_C}_0|$ can be bounded by $2^{e_{\max} - W + p + p - 1}$, the absolute value of all fields in an indexed type can be bounded by $2^{e_{\max} - W + p + p - 1}$.

    As the maximum useful $fold$ is $\floor{(e_{\max} - e_{\min} + p - 2)/W} + 1$,

    $\sum_{k = 0}^{2 \times fold - 1} |y_k| < e_{\max} - W + p + p - 1 + \ceil{\log_2{\floor{(e_{\max} - e_{\min} + p - 2)/W}}}$

    If the intermediate precision has a maximum exponent greater than or equal to $e_{\max} - W + p + p - 1 + \ceil{\log_2{\floor{(e_{\max} - e_{\min} + p - 2)/W}}}$, then no special cases to guard against overflow are needed.

    Algorithm \ref{alg:conv2float} represents a conversion routine in such a case.

    \begin{alg}
      Convert indexed $Y$ with index $i$ and fold $fold$ to floating point $x$. Here, $z$ is a floating point type with precision $P$ greater than $p + \ceil{\log_2(2 \times fold)}$ and maximum exponent $E_{\max}$ greater than $e_{\max} - W + p + p - 1 + \ceil{\log_2{\floor{(e_{\max} - e_{\min} + p - 2)/W}}}$
      \begin{algorithmic}
        \Function{ConvertIndexedToFloat}{fold, x, Y}
          \State $z \gets {y'_C}_0$
          \For{$j \in 1 \to fold - 1$}
            \State $z \gets z + {y'_C}_j$
            \State $z \gets z + {y'_P}_{j - 1}$
          \EndFor
          \State $z \gets {y'_P}_{fold - 1}$
          \State $x \gets z$
        \EndFunction
      \end{algorithmic}
      \label{alg:conv2float}
    \end{alg}

    If an intermediate type with exponent greater than $e_{\max} - W + p + p - 1 + \ceil{\log_2{\floor{(e_{\max} - e_{\min} + p - 2)/W}}}$ is not available, we must scale the intermediate sum appropriately.
    Although the fields have an intrinsic unnormalized ordering, this ordering does not necessarily guard against overflow. If the intermediate precision used for summation does not have a larger exponent range that the indexed precision (they must at least be equal), we must scale the top of the range down by some factor during addition, and then scale it back up when subsequent additions can no longer affect an overflow situation.

        $y = \sum\limits_{j = 0}^{fold - 1} {y_P}_j + {y_C}_j = \sum\limits_{j = 0}^{fold - 1} ({Y_P}_j - 1.5 \times\epsilon^{-1}\times 2^{a_{i + j}}) + {Y_C}_j \times 0.25\times\epsilon^{-1}\times2^{a_{i + j}}$

  \subsection{Error Bound}
  \subsection{Limits}
    \label{sec:limits}
    As absolute value of individual quantities added to ${Y_P}_j$ are not in excess of $2^W b_{i + j}$, $0.25\epsilon^{-1}2^{-W}$ elements may be deposited into ${Y_P}_j$ between renormalizations. As ${Y_C}_j$ must be able to record additions of absolute value 1 without error, ${Y_C}_j$ must stay in the range $(\epsilon^{-1}, -\epsilon^{-1})$. As each renormalization results in addition not in excess absolute value of 2 to ${Y_C}_j$, a maximum of $0.5 * (\epsilon^(-1) - 1)$ renormalizations may be performed, meaning that.
\section{Interface}
  \subsection{Fold}
  \subsection{Complex Types}
  \subsection{Naming Conventions}
  \subsection{Build System}
\section{indexed.h}
  \subsection{Overview}
  \subsection{Types}
  \subsection{Functions}
    \subsubsection{xixadd}
    \subsubsection{xixiadd}
    \subsubsection{xscale}
    \subsubsection{xixiaddsq}
\section{idxdBLAS.h}
  \subsection{Overview}
  \subsection{Functions}
    \subsubsection{xixdot}
    \subsubsection{xixnrm2}
    \subsubsection{xixgemv}
    \subsubsection{xixgemm}
  \subsection{Optimization}
    Due to the proportion of time spent in the deposit routine, optimization of the deposit routine was prioritized. In the event that multiple $x$ need to be added to $Y$, the deposit routine can be vectorized by accumulating the $x$ in copies of $Y$. The process is formalized in \ref{alg:depositvectorized}. The number of copies of $Y$ to make, $c$, is a tuning parameter.
      \begin{alg}
        Extract components of each $x_k$ in a floating point vector $\langle x_0, ..., x_{c - 1} \rangle$ in bins $(a_i, b_i], ..., (a_{i + fold - 1}, b_{i + fold - 1}]$ and add to indexed $Y$. Here, $(\vec{r} | 1)$ represents the result of setting the last bit of the mantissa ($m_{p - 1}$) of each floating-point $r_k$ to 1. Note that $\vec{Z}$ is used to represent a $c$-vector of values $\langle Z_0, ..., Z_{c - 1} \rangle$.
        \begin{algorithmic}
          \Function{DepositVectorized}{fold, $\vec{x}$, Y}
            \State $\vec{Z} \gets \langle Y, ..., Y \rangle$
            \State $\vec{r} \gets \vec{x}$
            \For{$i = 0 \to (fold - 2)$}
              \State $\vec{S} \gets {\vec{Z_P}}_i + (\vec{r} | 1)$
              \State $\vec{q} \gets \vec{S} - {\vec{Z_P}}_i$
              \State ${\vec{Z_P}}_i \gets \vec{S}$
              \State $\vec{r} \gets \vec{r} - \vec{q}$
            \EndFor
            \State ${\vec{Z_P}}_{fold - 1} \gets {\vec{Z_P}}_{fold - 1} + (\vec{r} | 1)$
            \State $Y_P \gets Y_P + \sum(\vec{Z_P} - \langle Y_P, ..., Y_P \rangle)$
          \EndFunction
        \end{algorithmic}
        \label{alg:depositvectorized}
      \end{alg}
      It is clear that before the last line of algorithm \ref{alg:depositvectorized}, each $Z_k \in \langle Z_0, ..., Z_{c - 1} \rangle$ represents the result of depositing (using algorithm \ref{alg:deposit}) $x_k$ in $Y$. Assuming that ${Y_P}_j$ is renormalized to the range $[1.5 \times $
\section{repBLAS.h}
\section{idxdMPI.h}
  \subsection{Types}
  \subsection{Functions}
    \subsubsection{XIXIREDUCE}
\section{Applications}
  \subsection{Parallel Reproducible Dot Product}
  \subsection{Parallel Reproducible Vector Norm}
  \subsection{Parallel Reproducible Matrix-Vector Multiply}
  \subsection{Parallel Reproducible Matrix-Matrix Multiply}
\begin{thebibliography}{9}
  \bibitem{repsum}
    Demmel, James, and Hong Diep Nguyen. Parallel Reproducible Summation. IEEE Transactions on Computers, 2014.
  \bibitem{ieee754}
    IEEE Standard for Floating-Point Arithmetic," IEEE Std 754-2008 , vol., no., pp.1,70, Aug. 29 2008
  \bibitem{c89}
    ANSI/ISO 9899-1990 American National Standard for Programming Languages - C, section 6.1.2.5
  \bibitem{sortsum}
    Demmel, James W., and Yozo Hida. Accurate floating point summation. Computer Science Division, University of California, 2002.
\end{thebibliography}
\end{document}
