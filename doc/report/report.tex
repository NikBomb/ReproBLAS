\documentclass[12pt]{article}
\usepackage{amsfonts}
\usepackage{listings}
\usepackage{amsmath}
\usepackage{amsthm}
\usepackage{algorithmicx}
\usepackage{algpseudocode}
\usepackage{graphicx}
\usepackage{tikz}
\pagestyle{myheadings}

\author{Peter Ahrens}
\title{ReproBLAS: Much Summation! So Reproducible!}
\providecommand{\ceil}[1]{\left \lceil #1 \right \rceil }
\providecommand{\floor}[1]{\left \lfloor #1 \right \rfloor }
\providecommand{\R}{\ensuremath{\mathbb{R}}}
\providecommand{\F}{\ensuremath{\mathbb{F}}}
\providecommand{\Z}{\ensuremath{\mathbb{Z}}}
\providecommand{\exp}{\ensuremath{\text{exp}}}
\providecommand{\min}{\ensuremath{\text{min}}}
\providecommand{\max}{\ensuremath{\text{max}}}
\providecommand{\ulp}{\ensuremath{\text{ulp}}}
\providecommand{\ufp}{\ensuremath{\text{ufp}}}
\theoremstyle{plain}
\newtheorem{thm}{Theorem}[section]
\newtheorem{lem}[thm]{Lemma}
\newtheorem{alg}{Algorithm}[section]
\graphicspath{{plots}}
\begin{document}
\noindent
\maketitle
\newpage
\section{Introduction}

\section{Notation and Background}
  Let $\R, \F, $ and $\Z$ denote the sets of real numbers, floating-point numbers, and integers respectively.

  Assume that floating point arithmetic complies with the IEEE 754-2008 standard in "to nearest even" rounding mode and that underflow occurs gradually.

  Let $f = s \times 2^e \times m \in \F$ be a floating-point number represented in IEEE 754 format where $s = \pm 1$ is the \textbf{sign}, $e_{\max} \geq e \geq e_{\min}$ is the \textbf{exponent} ($\exp(f)$ is defined to be $e$), $p$ is the \textbf{precision}, and $m = m_0.m_1m_2...m_{p-1}$ where $m_1, ..., m_{p - 1} \in \{0, 1\}$ is the \textbf{significand} of $f$. $f$ is said to be \textbf{normalized} if $m_0 = 1$ and $e > e_{\min}$, \textbf{unnormalized} if $m_0 = 0$, and \textbf{subnormal} if $m_0 = 0$ and $e = e_{\min}$. $f = 0$ if $m_0 = m_1 = ... = m_{p - 1} = 0$ and $e = e_{\min}$.

  As ReproBLAS is written in C, \verb|float| and \verb|double| refer to the floating point types specified in the 1989 C standard \cite{c89} and we assume that they correspond to the 32-bit and 64-bit types in the IEEE-754-2008 floating point standard \cite{ieee754}. 

  All indicies start at 0 to increase correspondence with the actual ReproBLAS implementation.

\section{The Indexed Type}
  \subsection{Representation}
    The \textbf{indexed type} is used to represent the intermediate result of accumulation using algorithms 6 and 7 specified by Nguyen and Demmel in "Parallel Reproducible Summation" \cite{repsum}.
    An indexed type $Y$ is composed of several accumulators $Y_0, ..., Y_{K - 1} \in \F$, each holding the sum of input in a predetermined exponent range (referred to as a \textbf{bin}). An indexed type with $K$ accumulators is referred to as having \textbf{fold} $K$.

    Let $Y$ be the indexed type corresponding to the reproducibly computed sum of $x_0, ..., x_{n - 1} = s_0 \times 2^{e_0} \times m_0, ..., s_{n - 1} \times 2^{e_{n - 1}} \times m_{n - 1} \in F$.
    If the accumulator $Y_j$ accumulates input in the bin $[a, b), a > b \in \Z$, then $Y_j$ is the sum of all input bits with exponents in $[a, b)$. Specifically, $Y_j = \sum \{s_i2^{e_i}{m_i}_k, i \in \{0, ..., n - 1\}, k \in \Z | e_i - k \in [a, b)\}$.

    For ease of computation, accumulator ranges are not arbitrary. They are broken up into precomputed, fixed width intervals starting at $e_{\max}$. The possible bins $[a_0, b_0), [a_1, b_1), ...$ for an indexed type of \textbf{width} $W$ are $[e_{\max}, e_{\max} - W), [e_{\max} - W, e_{\max} - 2W), ..., [e_{\min} - p - W - (e_{\max} - e_{\min} + p \mod W), e_{\min} - p - (e_{\max} - e_{\min} + p \mod W))$. It will be explained later why input in the bin $[e_{\min} - p + (e_{\max} - e_{\min} + p \mod W) - W, e_{\min} - p)$ is not accumulated.

    The accumulators in an indexed type correspond to contiguous bins in decreasing order. The \textbf{index} of $Y$ is the smallest $i \in \Z$ such that $2^(a_i + 1) > \max{|x_0|, ..., |x_n|}$ (equivalently, the smallest $i \in \Z$ such that $a_i \geq \max{e_0, ..., e_n}$). If $Y$ has index $i$, then $Y_j, j \in \{1, ..., K\}$ accumulates numbers in the bin $[a_{i + j}, b_{i + j})$.
  \subsection{Implementation}
    \subsubsection{Data Types}
      As discussed in \cite{repsum}, indexed types are represented using floating point numbers to minimize traffic between floating point and integer arithmetic units. If an indexed type is used to sum \verb|double|, then it is composed entirely of \verb|double| and likewise for \verb|float|. ReproBLAS supports complex types as pairs of real and imaginary components (stored contiguously in memory). If an indexed type is used to sum complex \verb|double| or \verb|float|, then it is composed of pairs of real and imaginary \verb|double| or \verb|float| respectively. The decision to keep the real and imaginary components together (as opposed to keeping separate indexed types for real and imaginary parts of the sum) was motivated by a desire to process accumulators simultaneously with vectorized (SIMD, single instruction multiple data) instructions.
    \subsubsection{Primary and Carry}
      An indexed type's accumulators are implemented using two underlying floating point fields. The \textbf{primary} field is used during accumulation, while the \textbf{carry} field holds overflow from the primary field. When discussing the primary and carry fields of the accumulator $Y$, we denote the primary fields by $Y_P$ and the carry fields by $Y_C$, so that ${Y_P}_j$ is the primary field of the accumulator $Y_j$. This reflects the fact that the primary fields are stored separately from the carry fields as the primary fields are frequenly accessed sequentially.

      The primary field ${Y_P}_j$ of an indexed type $Y$ of index $i$ is stored as an offset from $1.5 * \epsilon^{-1} 2^{b_{i + j}}$. The primary driving factor behind this decision is that it greatly simplifies the process of extracting the components of the input belonging in bin $[a_i, b_i)$. As discussed in \cite{repsum}, algorithm \ref{alg:deposit} extracts the components of $x$ belonging to bins $[a_i, b_i), ..., [a_{i + K - 1}, b_{i + K - 1})$ and adds it to ${Y_P}_0, ..., {Y_P}_{K - 1}$ for all $x \in \F, |x| < 2 * 2^{a_i}$.
      \begin{alg}
        Extract component of floating point $x$ in bin $[a_i, b_i)$ and add to $Y$.
        \begin{algorithmic}
          \Function{Deposit}{fold, Y, x}
            
          \EndFunction
        \end{algorithmic}
        \label{alg:extract}
      \end{alg}

\section{Indexed BLAS1}
\section{Indexed General Matrix-Vector Multiply}
\section{Indexed General Matrix-Matrix Multiply}
\section{Indexed MPI}
\begin{thebibliography}{9}
  \bibitem{sortingsummation}
    Demmel, James W., and Yozo Hida. Accurate floating point summation. Computer Science Division, University of California, 2002.
  \bibitem{repsum}
    Demmel, James, and Hong Diep Nguyen. Parallel Reproducible Summation. IEEE Transactions on Computers, 2014.
  \bibitem{ieee754}
    IEEE Standard for Floating-Point Arithmetic," IEEE Std 754-2008 , vol., no., pp.1,70, Aug. 29 2008
  \bibitem{c89}
    ANSI/ISO 9899-1990 American National Standard for Programming Languages - C, section 6.1.2.5
\end{thebibliography}
\end{document}
