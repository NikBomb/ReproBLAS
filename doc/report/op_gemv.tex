\subsection{Matrix-Vector Product}
  \label{sec:compositeops_gemv}
  The matrix-vector product $y \in \R^m$ of an $m \times n$ matrix $A$ and a vector $x \in \R^n$ (denoted by $y = Ax$ and computed in the BLAS by \texttt{xgemv} \cite{BLAS}) is defined as follows (where $A_{i, j}$ is the entry in the $i^{th}$ row and $j^{th}$ column, indexing from zero for consistency):
  \[
    y_i = \sum\limits_{j = 0}^{n - 1} A_{i, j}x_j
  \]
  Before we get into disucussions of reproducibility in this situation, we must first discuss what it means for a matrix-vector product to be reproducible.
  It is clear that computation of the matrix vector product is easily distributed along different values of $i$, as rows of $A$ and copies of $x$ can be distributed among $p$ processors so that entries of $y$ may be calculated independently.
  However, in the case where $n$ is quite large compared to $m/p$, it may be necessary to parallelize calculation of the sums of each $y_i$ so that the entirety of $x$ does not need to be communicated.
  As the local sums would be computed separately and then combined, different blocking patterns or reduction tree shapes could lead to different results.
  Although the reproducible reduction discussed in Section \ref{compositeops_reduce} would gurantee reproducibility with respect to reduction tree shape, it would not guarantee reproducibility with respect to blocking pattern. This stronger guarantee can be obtained if each sum $y_i$ is calculated as an indexed sum.
  It is for this reason that we need a local version of \texttt{xgemv} that returns a vector of indexed sums, the matrix-vector product with all sums calculated using indexed summation.

  A consequence of calculating the indexed matrix-vector product using indexed sums is that it will be reproducible with respect to permutation of the columns of $A$ together with entries of $x$.
  Let $\sigma_0, ..., \sigma_{n - 1}$ be some permutation of the first $n$ nonnegative integers such that $\{\sigma_0, ..., \sigma_{n - 1}\} = \{0, ..., n - 1\}$ as sets. Then we have
  \[
    y_i = \sum\limits_{j = 0}^{n - 1} A_{i, j}x_j = \sum\limits_{j = 0}^{n - 1} A_{i, \sigma_j}x_{\sigma_j}
  \]

  More imporantly, the matrix-vector product should remain reproducible under any reduction tree shape.
  If $A = [A_{(0)}, A_{(1)}]$ where $A_{(0)}$ and $A_{(1)}$ are submatricies of size $m \times n_{(0)}$ and $m \times n_{(1)}$ and if $x = [x_{(0)}, x_{(1)}]$ where $x_{(0)}$ and $x_{(1)}$ are subvectors of size $n_{(0)}$ and $n_{(1)}$ then we have 
  \[
    Ax = A_{(0)}x_{(0)} + A_{(1)}x_{(1)}
  \]

  It is clear from Theorems \ref{thm:indexed_sum_unique} and the ``Ensure'' claim of Algorithm \ref{alg:addindexedtoindexed} that if the matrix-vector product is computed using indexed summation, the necessary properties hold.

  Actually computing the matrix-vector product using indexed summation is not difficult. In ReproBLAS, we mirror the function definition of \text{xgemv} in the BLAS as closely as possible, adding two additional parameters $\alpha$ and $\beta$ so that the entire operation performs the update $y \gets \alpha Ax + \beta y$ (we also add the standard parameters for transposing the matrix and selecting ordering of the matrix).

  At the core, ReproBLAS provides the function \texttt{idxdBLAS\_xixgemv} in \texttt{idxdBLAS.h}. This version of the function adds to the vector of indexed sums $y$ the previously mentioned indexed matrix vector product of the floating point $A$ and $x$, where $x$ is first scaled by $\alpha$ as is done in the reference BLAS implementation \cite{netlib}. It is important to notice that the parameter $\beta$ is excluded, as we do not yet know how to scale indexed types by different values while maintaining reproducibility.
  It is of note that because the reproducible dot product is so compute-heavy, we can get good performance implementing the matrix-vector product using the \texttt{idxdBLAS\_diddot} routine at the core. We must use a rectangular blocking strategy to ensure good caching behavior, and because we are making calls to the dot product routine, it is sometimes (depending on the transpose and row-major vs. column-major parameters) necessary to transpose each block of $A$ before computing the dot product. These considerations ensure that the dot product can compute quickly on sequential sections of cached data.

  Built on top of \texttt{idxdBLAS\_xixgemv} is the routine \texttt{reproBLAS\_rxgemv} in \texttt{reproBLAS.h}, which takes as input floating point $A$, $x$, $y$, $\alpha$, and $\beta$. $y$ is scaled by $\beta$, then converted to a vector of indexed types. The matrix-vector product is computed using indexed summation with a user specified number of accumulators, and the output is then converted back to floating point and returned. The routine \texttt{reproBLAS\_xgemv} uses the default number of accumulators.
