\section{Notation and Background}
  Let $\R$ and $\Z$ denote the sets of real numbers and integers respectively.

  For all $r \in \R$, let $r\Z$ denote the set of all multiples of $r$,
  $\{rz | z \in \Z\}$.

  For all $r \in \R$, let $\lceil r \rceil$ be the minimum element $z \in \Z$
  such that $z \geq r$.

  For all $r \in \R$, let $\lfloor r \rfloor$ be the maximum element $z \in \Z$
  such that $z \leq r$.

  We define the function $\roundtonearestinfty(r, e), r \in R, e \in \Z$ as

  \begin{equation}
    \roundtonearestinfty(r, e) = \begin{cases}
        \lfloor r/2^e + 1/2 \rfloor 2^e \text{ if } r \geq 0\\
        \lceil r/2^e - 1/2 \rceil 2^e \text{ otherwise}
    \end{cases}
  \end{equation}

  $\roundtonearestinfty(r, e)$ rounds $r$ to the nearest multiple of $2^e$,
  breaking ties away from 0. Properties of such rounding are shown in
  \eqref{eq:round}
  \begin{equation}
    \begin{aligned}
    \bigl|r - \roundtonearestinfty(r, e)\bigr| & \leq 2^{e - 1} \\
    \roundtonearestinfty(r,e) & = 0 \text{ if } r < 2^{e-1}.
    \end{aligned}
    \label{eq:round}
  \end{equation}

  Let $\F_{b,p,e_{min},e_{max}}$ denote the set of floating-point numbers
  of base $b \in \Z$ ($b > 1$),
  precision $p \in Z$ ($p > 0$) and exponent range $[e_{min}, e_{max}]$
  where $e_{min}, e_{max} \in \Z$ and $e_{min} \leq e_{max}$.
  Each value $f \in \F_{b,p,e_{min},e_{max}}$ is represented by:
  \[
    f = s \cdot m_0.m_1 \ldots m_{p-1} \cdot b^e,
  \]
  where $s \in \{-1,1 \}$ is the sign,
  $e \in \{\Z | e_{\min} \leq e \leq e_{\max} \}$ is the exponent
  (also defined as $exp(f)$),
  and $m = m_0.m_1 \ldots m_{p-1}, m_i \in \{ 0, 1, \ldots, b-1 \}$
  is the significand (also called the mantissa) of $f$.

  Although many of the below analysis can be applied to a general floating-point
  format, in the context of this paper we assume binary floating-point formats,
  i.e. $b=2$, complying with the IEEE 754-2008 standard \cite{ieee754}
  in single ($p=24$) and double ($p=53$) precision which are encoded in hardware
  using words of 32 and 64 bits respectively.
  Specific parameters for those two formats are given in table~\ref{tbl:IEEE-754}.
  \begin{table}
    \caption{IEEE 754-2008 binary floating-point formats}
    \label{tbl:IEEE-754}
        \centering
        \begin{tabular}{ | l | l | l | } \hline
            Floating-Point Type & single precision & double precision \\ \hline
            C data type & \texttt{float} & \texttt{double} \\ \hline
            $p$ & 24 & 53 \\ \hline
            Exponent field width & 8 & 12 \\ \hline
            Exponent bias & 127 & 1023 \\ \hline
            $e_{min}$ & -126 & -1022 \\ \hline
            $e_{max}$ & 127 & 1023 \\ \hline
        \end{tabular}
  \end{table}

  \begin{comment}
  Let $\F$ be the set of all floating-point numbers $f = sm2^e$ represented in some binary IEEE 754-2008 format \cite{ieee754} where $s \in \{1, -1\}$ is the \textbf{sign},
  $e \in \Z$, $e_{\max} \geq e \geq e_{\min}$ is the \textbf{exponent} ($\exp(f)$ is
  defined to be $e$), $p \in \Z, p > 1$ is the \textbf{precision},
  and $m=m_0.m_1m_2...m_{p-1}$ where $m_0, ..., m_{p - 1} \in \{0, 1\}$ is the
  \textbf{significand} of $f$. Assume that the representation of $f$ is made unique using the ``hidden bit'' convention, so that $f$ is represented using the smallest exponent possible. (In memory, the bit $m_0$ is not stored and assumed to be 1 unless the exponent field contains a special value signaling both that $e = e_{\min}$ and $m_0=0$).
  $f$ is said to be \textbf{normalized} if $m_0 =1$
  and $e \geq e_{\min}$, \textbf{unnormalized} if $m_0 = 0$, and
  \textbf{denormalized} if $m_0 = 0$ and $e = e_{\min}$. $f = 0$ if all $m_j = 0$
  and $e = e_{\min}$.

  Assume that floating point arithmetic complies with the IEEE 754-2008
  standard \cite{ieee754} in some ``to nearest'' rounding mode (no specific tie
  breaking behaviour is required) and that underflow occurs gradually, although
  methods to handle abrupt underflow will be considered in Section
  \ref{sec:indexed_underflow_abrupt}.

  Let $f = sm2^e \in \F$ be a floating-point number represented in IEEE
  754-2008 format \cite{ieee754} where $s \in \{1, -1\}$ is the \textbf{sign},
  $e_{\max} \geq e \geq e_{\min}$ is the \textbf{exponent} ($\exp(f)$ is
  defined to be $e$), $p$ is the \textbf{precision},
  and $m=m_0.m_1m_2...m_{p-1}$ where $m_0, ..., m_{p - 1} \in \{0, 1\}$ is the
  \textbf{significand} of $f$.
  \end{comment}

  The exponent is stored in internal representation in biased form
  using 8 bits for single precision, and 53 bits for double precision.
  The exponent of a represented floating-point number is equal to
  the biased exponent (the unsigned integer value stored in the exponent field)
  minus the bias value (see table~\ref{tbl:IEEE-754}).
  An exponent field of all 0-bit is reserved for zeros and denormalized values.
  An exponent field of all 1-bit is reserved for infinities and NaN.

  $f$ is said to be \textbf{normalized} if $m_0 =1$
  and $e_{max} \geq e \geq e_{\min}$.
  The first bit $m_0$ is not explicitly stored in internal representation
  and is called hidden or implicit bit.
  Therefore only $p-1$ bits are used to represent the mantissa of $f$.
  $f$ is said to be \textbf{unnormalized} if $m_0 = 0$, and
  \textbf{denormalized} if $m_0 = 0$ and $e = e_{\min} - 1$.
  $f = 0$ if all $m_j = 0$ and $e = e_{\min} - 1$.

  In binary internal representation, $f=0$ is represented by a biased exponent of $0$
  as well as a mantissa field of all 0-bits.
  Denormalized numbers have biased exponent of $0$ and non-zero mantissa field.
  An exponent field of all 1-bits and a mantissa field of all 0-bits
  represent infinities, positive or negative depending on the sign bit.
  An exponent field of all 1-bits and a non-zero mantissa field
  represent a NaN value.

  We assume rounding mode ``to nearest'' (no specific tie
  breaking behavior is required) and gradual underflow, although
  methods to handle abrupt underflow will be considered in Section
  \ref{sec:indexed_underflow_abrupt}.

  For simplicity as well as for readability, throughout this paper
  $\F_{b,p,e_{min},e_{max}}$ will be written simply as $\F$, referring to either
  the IEEE 754-2008 single or double precision binary floating-point format,
  i.e. $b=2$ and $m_i \in \{0, 1\}$.
  All the analysis will be based on the corresponding parameters $p$, $e_{min}$
  and $e_{max}$.

  $r \in \R$ is \textbf{representable} as a floating point number if there
  exists $f \in \F$ such that $r = f$ as real numbers.

  For all $r \in \R$, $e \in \Z$ such that $e_{\min} - p < e$ and $|r| < 2
  \cdot 2^{e_{\max}}$, if $r \in 2^e\Z$ and $|r| \leq 2^{e + p}$ then $r$ is
  representable.

  Machine epsilon, $\epsilon$, the difference between 1 and the greatest
  floating point number smaller than 1, is defined as $\epsilon = 2^{-p}$.

  The unit in the last place of $f \in \F$, $\ulp(f)$, is the spacing between
  two consecutive floating point numbers of the same exponent as $f$. If $f$ is
  normalized, $\ulp(f) = 2^{\exp(f) - p + 1} = 2  \epsilon  2^{\exp(f)}$ and
  $\ulp(f) \leq 2^{p - 1}|f|$.

  The unit in the first place of $f \in F$, $\ufp(f)$, is the value of the
  first significant bit of $f$. If $f$ is normalized, $\ufp(f) = 2^{\exp(f)}$.

  For all $f_0, f_1 \in \F$, $\fl(f_0 \text{ op } f_1)$ denotes the evaluated
  result of the expression $(f_0 \text{ op } f_1)$ in floating point
  arithmetic. If $(f_0 \text{ op } f_1)$ is representable, then
  \(
    \fl(f_0 \text{ op } f_1) = (f_0 \text{ op } f_1).
  \)
  If rounding is ``to nearest,'' then we have that
  \(
    |\fl(f_0 \text{ op } f_1) - (f_0 \text{ op } f_1)| \leq 0.5\ulp(\fl(f_0 \text{ op } f_1)).
  \)

  As ReproBLAS is written in C, \texttt{float} and \texttt{double} refer to the
  floating point types specified in the 1989 C standard \cite{c89} and we
  assume that they correspond to the \texttt{binary-32} and \texttt{binary-64}
  types in the IEEE 754-2008 floating point standard \cite{ieee754}.

  All indices start at $0$ in correspondence with the actual ReproBLAS implementation.

