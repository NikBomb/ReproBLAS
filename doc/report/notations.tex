\section{Notation and Background}
  Let $\R$ and $\Z$ denote the sets of real numbers and integers respectively.

  For all $r \in \R$, let $r\Z$ denote the set of all multiples of $r$,
  $\{rz | z \in \Z\}$.

  For all $r \in \R$, let $\lceil r \rceil$ be the minimum element $z \in \Z$
  such that $z \geq r$.

  For all $r \in \R$, let $\lfloor r \rfloor$ be the maximum element $z \in \Z$
  such that $z \leq r$.

  We define the function $\roundtonearestinfty(r, e), r \in R, e \in \Z$ as

  \begin{equation}
    \roundtonearestinfty(r, e) = \begin{cases}
        \lfloor r/2^e + 1/2 \rfloor 2^e \text{ if } r \geq 0\\
        \lceil r/2^e - 1/2 \rceil 2^e \text{ otherwise}
    \end{cases}
  \end{equation}

  $\roundtonearestinfty(r, e)$ rounds $r$ to the nearest multiple of $2^e$,
  breaking ties away from 0. Properties of such rounding are shown in
  \eqref{eq:round}
  \begin{equation}
    \begin{aligned}
    \bigl|r - \roundtonearestinfty(r, e)\bigr| & \leq 2^{e - 1} \\
    \roundtonearestinfty(r,e) & = 0 \text{ if } r < 2^{e-1}.
    \end{aligned}
    \label{eq:round}
  \end{equation}

  Let $\F_{b,p,e_{min},e_{max}}$ denote the set of floating-point numbers
  of base $b \in \Z$ ($b > 1$),
  precision $p \in Z$ ($p > 0$) and exponent range $[e_{min}, e_{max}]$
  where $e_{min}, e_{max} \in \Z$ and $e_{min} \leq e_{max}$.
  Each value $f \in \F_{b,p,e_{min},e_{max}}$ is represented by:
  \[
    f = s \cdot m_0.m_1 \ldots m_{p-1} \cdot b^e,
  \]
  where $s \in \{-1,1 \}$ is the sign,
  $e \in \{\Z | e_{\min} \leq e \leq e_{\max} \}$ is the exponent
  (also defined as $exp(f)$),
  and $m = m_0.m_1 \ldots m_{p-1}, m_i \in \{ 0, 1, \ldots, b-1 \}$
  is the significand (also called the mantissa) of $f$.

  Although many of the below analysis can be applied to a general floating-point
  format, in the context of this paper we assume binary floating-point formats,
  i.e. $b=2$, complying with the IEEE 754-2008 standard \cite{ieee754}
  in single ($p=24$) and double ($p=53$) precision which are encoded in hardware
  using words of 32 and 64 bits respectively.
  Specific parameters for those two formats are given in table~\ref{tbl:IEEE-754}.
  \begin{table}
    \caption{IEEE 754-2008 binary floating-point formats}
    \label{tbl:IEEE-754}
        \centering
        \begin{tabular}{ | l | l | l | } \hline
            Floating-Point Type & single precision & double precision \\ \hline
            C data type & \texttt{float} & \texttt{double} \\ \hline
            $p$ & 24 & 53 \\ \hline
            Exponent field width & 8 & 12 \\ \hline
            Exponent bias & 127 & 1023 \\ \hline
            $e_{min}$ & -126 & -1022 \\ \hline
            $e_{max}$ & 127 & 1023 \\ \hline
        \end{tabular}
  \end{table}

  \begin{comment}
  Assume that floating point arithmetic complies with the IEEE 754-2008
  standard \cite{ieee754} in some ``to nearest'' rounding mode (no specific tie
  breaking behaviour is required) and that underflow occurs gradually, although
  methods to handle abrupt underflow will be considered in Section
  \ref{sec:indexed_underflow_abrupt}.

  Let $f = sm2^e \in \F$ be a floating-point number represented in IEEE
  754-2008 format \cite{ieee754} where $s \in \{1, -1\}$ is the \textbf{sign},
  $e_{\max} \geq e \geq e_{\min}$ is the \textbf{exponent} ($\exp(f)$ is
  defined to be $e$), $p$ is the \textbf{precision},
  and $m=m_0.m_1m_2...m_{p-1}$ where $m_0, ..., m_{p - 1} \in \{0, 1\}$ is the
  \textbf{significand} of $f$.
  \end{comment}

  The exponent is stored in internal representation in biased form
  using 8 bits for single precision, and 53 bits for double precision.
  The exponent of a represented floating-point number is equal to
  the biased exponent (the unsigned integer value stored in the exponent field)
  minus the bias value (see table~\ref{tbl:IEEE-754}).
  An exponent field of all 0-bit is reserved for zeros and denormalized values.
  An exponent field of all 1-bit is reserved for infinities and NaN.

  $f$ is said to be \textbf{normalized} if $m_0 =1$
  and $e_{max} \geq e \geq e_{\min}$.
  The first bit $m_0$ is not explicitly stored in internal representation
  and is called hidden or implicit bit.
  Therefore only $p-1$ bits are used to represent the mantissa of $f$.
  $f$ is said to be \textbf{unnormalized} if $m_0 = 0$, and
  \textbf{denormalized} if $m_0 = 0$ and $e = e_{\min} - 1$.
  $f = 0$ if all $m_j = 0$ and $e = e_{\min} - 1$.

  In binary internal representation, $f=0$ is represented by a biased exponent of $0$
  as well as a mantissa field of all 0-bits.
  Denormalized numbers have biased exponent of $0$ and non-zero mantissa field.
  An exponent field of all 1-bits and a mantissa field of all 0-bits
  represent infinities, positive or negative depending on the sign bit.
  An exponent field of all 1-bits and a non-zero mantissa field
  represent a NaN value.

  We assume rounding mode ``to nearest'' (no specific tie
  breaking behavior is required) and gradual underflow, although
  methods to handle abrupt underflow will be considered in Section
  \ref{sec:indexed_underflow_abrupt}.

  For simplicity as well as for readability, throughout this paper
  $\F_{b,p,e_{min},e_{max}}$ will be written simply as $\F$, referring to either
  the IEEE 754-2008 single or double precision binary floating-point format,
  i.e. $b=2$ and $m_i \in \{0, 1\}$.
  All the analysis will be based on the corresponding parameters $p$, $e_{min}$
  and $e_{max}$.

  $r \in \R$ is \textbf{representable} as a floating point number if there
  exists $f \in \F$ such that $r = f$ as real numbers.

  For all $r \in \R$, $e \in \Z$ such that $e_{\min} - p < e$ and $|r| < 2
  \cdot 2^{e_{\max}}$, if $r \in 2^e\Z$ and $|r| \leq 2^{e + p}$ then $r$ is
  representable.

  Machine epsilon, $\epsilon$, the difference between 1 and the greatest
  floating point number smaller than 1, is defined as $\epsilon = 2^{-p}$.

  The unit in the last place of $f \in \F$, $\ulp(f)$, is the spacing between
  two consecutive floating point numbers of the same exponent as $f$. If $f$ is
  normalized, $\ulp(f) = 2^{\exp(f) - p + 1} = 2  \epsilon  2^{\exp(f)}$ and
  $\ulp(f) \leq 2^{p - 1}|f|$.

  The unit in the first place of $f \in F$, $\ufp(f)$, is the value of the
  first significant bit of $f$. If $f$ is normalized, $\ufp(f) = 2^{\exp(f)}$.

  For all $f_0, f_1 \in \F$, $\fl(f_0 \text{ op } f_1)$ denotes the evaluated
  result of the expression $(f_0 \text{ op } f_1)$ in floating point
  arithmetic. If $(f_0 \text{ op } f_1)$ is representable, then
  \(
    \fl(f_0 \text{ op } f_1) = (f_0 \text{ op } f_1).
  \)
  If rounding is ``to nearest,'' then we have that
  \(
    |\fl(f_0 \text{ op } f_1) - (f_0 \text{ op } f_1)| \leq 0.5\ulp(\fl(f_0 \text{ op } f_1)).
  \)

  As ReproBLAS is written in C, \texttt{float} and \texttt{double} refer to the
  floating point types specified in the 1989 C standard \cite{c89} and we
  assume that they correspond to the \texttt{binary-32} and \texttt{binary-64}
  types in the IEEE 754-2008 floating point standard \cite{ieee754}.

  All indices start at $0$ in correspondence with the actual ReproBLAS implementation.

\section{Binning}
\label{sec:binning}
We achieve reproducible summation of floating point numbers through binning.
Each number is split into several components corresponding to predefined
exponent ranges, then the components corresponding to each range are
accumulated separately. We begin in Section \ref{sec:binning_bins} by
explaining the particular set of ranges (referred to as bins) used. Section
\ref{sec:binning_slices} develops mathematical theory to describe the
components (referred to as slices) corresponding to each bin. We develop this
theory to concisely describe and prove correctness of algorithms throughout the
paper (especially Algorithms \ref{alg:depositrestricted} and
\ref{alg:deposit}).

    \subsection{Bins}
    \label{sec:binning_bins}
    We start by dividing the exponent range $(e_{\min} - p, ..., e_{\max} + 1]$
    into \textbf{bins} $(a_i, b_i]$ of \textbf{width} $W$ according to
    \eqref{eq:imax}, \eqref{eq:a}, and \eqref{eq:b}. Such a range is used so
    that the largest and smallest (denormalized) floating point numbers may be
    approximated.
    \begin{align}
        i_{\max} & = \bigl\lfloor(e_{\max} - e_{\min} + p - 1)/W\bigr\rfloor - 1
            \label{eq:imax} \\
        a_i & = e_{\max} + 1 - (i + 1)W \text{ for } 0 \leq i \leq i_{\max}
            \label{eq:a} \\
        b_i & = a_i + W
            \label{eq:b}
    \end{align}

    We say the bin $(a_{i_0}, b_{i_0}]$ is \textbf{greater} than the bin
    $(a_{i_1}, b_{i_1}]$ if $a_{i_0} > a_{i_1}$ (which is equivalent to both
    $b_{i_0} > b_{i_1}$ and $i_0 < i_1$).

    We say the bin $(a_{i_0}, b_{i_0}]$ is \textbf{less} than the bin
    $(a_{i_1}, b_{i_1}]$ if $a_{i_0} < a_{i_1}$ (which is equivalent to both
    $b_{i_0} < b_{i_1}$ and $i_0 > i_1$).

    We use $i \leq i_{\max} = \lfloor(e_{\max} - e_{\min} + p - 1)/W\rfloor - 1$
    to ensure that $a_i > e_{\min} - p + 1$ as discussed in Section
    \ref{sec:indexed_underflow_gradual}. This means that the greatest bin,
    $(a_{0}, b_{0}]$, is
    \begin{equation}
      (e_{\max} + 1 - W, e_{\max} + 1]
      \label{eq:binmax}
    \end{equation}

    and the least bin, $(a_{i_{\max}}, b_{i_{\max}}]$, is
    \begin{equation}
      \Bigl(e_{\min} - p + 2 + \bigl((e_{\max} - e_{\min} + p - 1)\mod W\bigr),
      e_{\min} - p + 2 + W + \bigl((e_{\max} - e_{\min} + p - 1)\mod W\bigr)\Bigr]
      \label{eq:binmin}
    \end{equation}

    Section \ref{sec:indexed_underflow_gradual} explains why the bottom of the exponent range
    \begin{equation*}
    \Bigl(e_{\min} - p, e_{\min} - p + 2 + \bigl((e_{\max} - e_{\min} + p - 1) \mod W\bigr)\Bigr]
    \end{equation*}
    is ignored.

    As discussed in \cite{repsum}, and explained again in Section~\ref{sec:primitiveops_renormalize},
    we must assume
    \begin{equation}
      W < p - 2
      \label{eq:wupper}
    \end{equation}

    As discussed in Section \ref{sec:indexed_overflow}, we must also assume
    \begin{equation}
      2 W > p + 1
      \label{eq:wlower}
    \end{equation}

    ReproBLAS uses both \texttt{float} and \texttt{double} floating point
    types. The chosen division of exponent ranges for both types is shown in
    Figure \ref{fig:bins}.

    \begin{figure}[H]
        \centering
        \begin{tabular}{ | l | l | l | p{5cm} |} \hline
            Floating-Point Type & \texttt{float} & \texttt{double}\\ \hline
            $e_{\max}$ & 127 & 1023\\ \hline
            $e_{\min}$ &  -126 & -1022 \\ \hline
            $p$ & 24 & 53 \\ \hline
            $e_{\min} - p$ & -140 & -1075 \\ \hline
            $W$ & 13 & 40 \\ \hline
            $i_{\max}$ & 19 & 51 \\ \hline
            $(a_0, b_0]$ & $(115, 128]$ & $(984, 1024]$\\ \hline
            $(a_{i_{\max}}, b_{i_{\max}}]$ & $(-132, -119]$ & $(-1056, -1016]$ \\ \hline
        \end{tabular}
        \caption{ReproBLAS Binning Scheme}
        \label{fig:bins}
    \end{figure}

    \subsection{Slices}
    \label{sec:binning_slices}
    Throughout the text we will refer to the \textbf{slice} of some $x \in \F$
    in the bin $(a_i, b_i]$. $x$ can be split into several slices, each slice
    corresponding to a bin $(a_i, b_i]$ and expressible as the (possibly
    negated) sum of a subset of $\{2^e, e \in (a_i, b_i]\}$, such that the sum
    of the slices provides a good approximation of $x$. Specifically, the slice
    of $x \in \F$ in the bin $(a_i, b_i]$ is defined recursively as $d(x, i)$
    in \eqref{eq:d}. We must define $d(x, i)$ recursively because it is not a
    simple bitwise extraction.
    \begin{equation}
      \begin{aligned}
      d(x,0) & = \roundtonearestinfty(x, a_0+1) \\
      d(x, i) & = \roundtonearestinfty\bigl(x - \sum\limits_{j=0}^{i - 1}d(x,j), a_i + 1\bigr)
        \text{ for } i > 0.
      \end{aligned}
      \label{eq:d}
    \end{equation}

    We make three initial observations on the definition of $d(x, i)$. First,
    we note that $d(x, i)$ is well defined recursively on $i$ with base case
    $d(x, 0) = \roundtonearestinfty(x, a_0 + 1)$.

    Next, notice that $d(x, i) \in 2^{a_{i} + 1}\Z$.

    Finally, it is possible that $d(x, 0)$ may be too large to represent as a
    floating point number for example, if $x$ is the largest finite floating point
    number then $d(x,0)=\roundtonearestinfty(x, a_0+1)$ would be $2^{e_{max}+1}$.
    Overflow of this type is accounted for in Section \ref{sec:indexed_overflow}.
    Technical detail of how to handle this special case during the binning process
    will be explained in Section~\ref{sec:primitiveops_deposit}.

    Lemmas \ref{lem:dzero} and \ref{lem:dmiddle} follow from the definition of $d(x, i)$.

    \begin{samepage}
    \begin{lem}
      For all $i \in \{0, ..., i_{\max}\}$ and $x \in \F$ such that $|x| < 2^{a_i}$,
      $d(x, i) = 0.$
      \label{lem:dzero}
    \end{lem}
    \end{samepage}

    \begin{proof}
      We show the claim by induction on $i$.

      In the base case, $|x| < 2^{a_0}$, by \eqref{eq:round} we have
      $d(x, 0) = \roundtonearestinfty(x, a_0 + 1) = 0$.

      In the inductive step, we have $|x| < 2^{a_{i + 1}} < \ldots < 2^{a_0}$ by \eqref{eq:a}
      and by induction $d(x, i)= ... = d(x, 0) = 0$. Thus,
      \[
        d(x, i + 1) = \roundtonearestinfty\bigl(x - \sum\limits_{j = 0}^{i}d(x, j), a_{i + 1} + 1\bigr)
            = \roundtonearestinfty(x, a_{i+1} + 1)
      \]
      Again, since $x < 2^{a_{i+1}}$, by \eqref{eq:round} we have
      \(
        d(x, i + 1) = \roundtonearestinfty(x, a_{i + 1} + 1) = 0.
      \)
    \end{proof}

    \begin{samepage}
    \begin{lem}
      For all $i \in \{0, ..., i_{\max}\}$ and $x \in \F$ such that $|x| < 2^{b_i}$,
      $d(x, i) = \roundtonearestinfty(x, a_i + 1)$.
      \label{lem:dmiddle}
    \end{lem}
    \end{samepage}

    \begin{proof}
      The claim is a simple consequence of Lemma \ref{lem:dzero}.

      By  \eqref{eq:a} and \eqref{eq:b}, $|x| < 2^{b_i} = 2^{a_{i - 1}} < \ldots <2^{a_0}$.
      Therefore Lemma \ref{lem:dzero} implies $d(x, 0) = ... = d(x, i - 1) = 0$
      and we have
      \[
        d(x, i) = \roundtonearestinfty\bigl(x - \sum\limits_{j = 0}^{i - 1}d(x, j), a_{i} + 1\bigr)
            = \roundtonearestinfty(x, a_{i} + 1)
      \]
    \end{proof}

    Lemma \ref{lem:dzero}, Lemma \ref{lem:dmiddle}, and \eqref{eq:d} can be
    combined to yield an equivalent definition of $d(x, i)$ for all $i \in \{0,
    ..., i_{\max}\}$ and $x \in \F$.

    \begin{equation}
      d(x, i) = \begin{cases}
        0 \text{ if } |x| < 2^{a_i}\\
        \roundtonearestinfty(x, a_i + 1) \text{ if } 2^{a_i} \leq |x| < 2^{b_i}\\
        \roundtonearestinfty\bigl(x - \sum\limits_{j=0}^{i - 1}d(x,j), a_i + 1\bigr) \text{ if } 2^{b_i} \leq |x|
        \end{cases}
      \label{eq:d2}
    \end{equation}

    Theorem \ref{thm:dround} shows that sum of the slices of $x \in \F$
    provides a good approximation of $x$.

    \begin{samepage}
    \begin{thm}
      For all $i \in \{0, ..., i_{\max}\}$ and $x \in \F$,
      $|x - \sum \limits_{j = 0}^id(x, j)| \leq 2^{a_i}$.
      \label{thm:dround}
    \end{thm}
    \end{samepage}

    \begin{proof}
      We apply  \eqref{eq:round} and \eqref{eq:d2}
      \begin{align*}
        \bigl|x - \sum \limits_{j = 0}^{i}d(x, j)\bigr| & = \Bigl|\bigl(x - \sum \limits_{j = 0}^{i - 1}d(x, j)\bigr) - d(x, i)\Bigr| \\
         & = \Bigl|\bigl(x - \sum \limits_{j = 0}^{i - 1}d(x, j)\bigr) - \roundtonearestinfty\bigl(x - \sum \limits_{j = 0}^{i - 1}d(x, j), a_{i} + 1\bigr)\Bigr| \leq 2^{a_{i}}
      \end{align*}
    \end{proof}

    Theorem \ref{thm:dbound} shows a bound on $d(x, i)$.

    \begin{samepage}
    \begin{thm}
      For all $i \in \{0, ..., i_{\max}\}$ and $x \in \F$, $|d(x, i)| \leq 2^{b_i}$.
      \label{thm:dbound}
    \end{thm}
    \end{samepage}

    \begin{proof}
      First, we show that $|x - \sum\limits_{j=0}^{i - 1}d(x,j)| \leq 2^{b_i}$.

      If $i = 0$, then we have
      \begin{equation*}
        \bigl|x - \sum\limits_{j=0}^{i - 1}d(x,j)\bigr| = |x| < 2 \cdot 2^{e_{\max}} < 2^{b_0}
      \end{equation*}
      Otherwise, we can apply  \eqref{eq:a} and \eqref{eq:b} to Theorem \ref{thm:dround} to get
      \begin{equation*}
        \bigl|x - \sum \limits_{j = 0}^{i - 1}d(x, j)\bigr| \leq 2^{a_{i - 1}} = 2^{b_i}
      \end{equation*}

      As $2^{b_i} \in 2^{a_i + 1}\Z$,  \eqref{eq:d} can be used

      \begin{equation*}
        \bigl|d(x, i)\bigr| = \Bigl|\roundtonearestinfty\bigl(x - \sum\limits_{j=0}^{i - 1}d(x,j), a_i + 1\bigr)\Bigr| \leq 2^{b_i}
      \end{equation*}
    \end{proof}

